\chapter[Legislação]{Legislação}

\section{Registro do Equipamento na ANVISA}

Segundo a Agência Nacional de Vigilância Sanitária (ANVISA), a Farmacopeia Brasileira é o Código Oficial Farmacêutico do país, o qual estabelece especificações e requisitos mínimos de qualidade para fármacos, insumos, drogas vegetais, medicamentos, produtos para a saúde e dispositivos médicos \cite{farmacopeia}. O cumprimento dos parâmetros farmacopeicos, quer seja para um artigo produzido em larga escala ou em manipulação individualizada, pode ser o fator preponderante para o resultado positivo de uma terapia racional \cite{Pianetti_2016}.

Assim, para obter a Certificação do Produto para fins de reconhecimento do equipamento no âmbito do Sistema Brasileiro de Avaliação da Conformidade (SBAC), é necessário seguir as Resoluções da Diretoria Colegiada e Instruções Normativas, conforme a carência do equipamento.

Com base na Resolução da Diretoria Colegiada - RDC  Nº 185 de 22 de outubro de 2001, a qual trata do registro, alteração, revalidação e cancelamento do registro de produtos médicos na ANVISA \cite{RDC_185}, é possível enquadrar produtos médicos em um conjunto de 4 classes, conforme o risco associado na utilização do equipamento.

Dessa forma, o produto médico ativo \textit{PillWatcher}, o qual é  destinado a administrar medicamentos, foi inserido na Regra 11 e na Classe de Risco II, que são produtos médicos ativos destinados a administrar medicamentos, conforme o anexo \ref{ane:RDC}. Além do mais, de acordo a Instrução Normativa - IN N° 13 de 22 de outubro de 2009, a qual dispõe sobre a documentação para regularização de equipamentos médicos das Classes de Risco I e II, foi possível realizar a ficha técnica do equipamento, o qual é presentado no apêndice \ref{manual_usuario}.

A Resolução RDC nº 249, de 13 de setembro de 2005 do Regulamento Técnico das Boas Práticas de Fabricação de Produtos Intermediários e Insumos Farmacêuticos Ativos \cite{RDC_249}, garante que os fabricantes assegurem que os produtos sejam adequados para o uso pretendido e que estejam de acordo com os requisitos de identidade, pureza e segurança, baseando-se nas políticas de qualidade pré-estabelecidas na Resolução RDC nº 304, de 17 de setembro de 2019.

\section{Segurança no SDMDU}

O Sistema de Distribuição de Medicamentos por Dose Unitária (SDMDU), utilizado no projeto \textit{PillWatcher}, garante maior segurança e eficiência, pois não requer manipulação do medicamento pelo enfermeiro e permite o acompanhamento farmacoterápico do usuário, diminuindo erros associados. Assim, os dados de identificação (nome do produto, concentração do princípio ativo, nº de registro, lote, prazo de validade etc.) deverão constar na unidade individualizada do medicamento.

Entretanto, o processo de unitizar as doses de suas embalagens primárias para uma embalagem secundária pode pôr a qualidade dos medicamentos em risco, comprometendo sua estabilidade e sua validade. Ao retirar do blister os comprimidos, cápsulas e drágeas, e os submeter a uma nova embalagem, a droga fica exposta ao material utilizado. Por isso, dois fatores importantes devem ser considerados: umidade (que pode causar a desestabilização química, caso o fármaco seja sensível a hidrólise e oxidação) e aumento da temperatura (que aumentará a permeabilidade dos gases através da embalagem) \cite{Jara_2012}. Assim, para realizar o processo de unitização, é necessário a utilização de luvas (troca a cada produto), máscaras e recipientes (bandejas plásticas/aço) cobertos com compressas previamente esterilizadas \cite{Everton_2012}. 

Dessa forma, para garantir a integridade do medicamento no processo de remoção do blister, foi criada a RDC Nº 80, de 11 de maio de 2006, a qual dispõe boas práticas para o processo de fracionamento do medicamento em farmácias e drogarias. Contudo, esse regulamento técnico não se aplica aos estabelecimentos de atendimento privativo de unidade hospitalar ou de qualquer outra equivalente de assistência médica. Da mesma forma, tem-se também o Decreto nº 5.775, de 10 de maio de 2006, o qual dispõe sobre o fracionamento de medicamentos. Além do mais, o anexo IV da RDC Nº 67/2007 estabelece os requisitos do preparo de doses unitária e unitização de doses de medicamentos em serviços de saúde.

No Brasil, há falta de políticas que regulamentem o uso dos medicamentos, e tem-se a incapacidade dos órgãos públicos em gerenciar informações acerca das reações adversas aos medicamentos \cite{Jara_2012}. Assim, estão sendo propostas novas normas para as instituições que necessitam realizar o processo de remoção do blister, as quais devem utilizar as normas do setor farmacêutico industrial, normas de fracionamento de medicamentos de farmácia e drogarias, normas de manipulação de medicamentos, entre outras, que regulam o setor farmacêutico, para assegurar as Boas Práticas Farmacêuticas aos produtos oferecidos aos usuários. 




