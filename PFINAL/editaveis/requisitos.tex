\chapter[Requisitos]{Requisitos}

Foi conduzido um teste piloto através de um questionário para médicos geriatras, cuidadores e enfermeiros de instituições de longa permanência para idosos disponível no apêndice \ref{questionario_app} e uma entrevista com um farmacêutico no apêndice \ref{entrevista_app}. Sendo assim, os questionários foram encaminhados via Internet com o objetivo de selecionar as necessidades do cliente. 

\section{Requisitos Gerais do Sistema}

Com o disposto nos questionários, elaboraram-se os requisitos gerais constantes na Tab. \ref{tab:req_gerais}.
\begin{table}[H]
    \centering
    \caption{Requisitos Gerais}
    \begin{tabular}{|L{2cm}|L{8cm}|C{2.5cm}|}
        \hline
        \rowcolor[HTML]{A8DADC}
        \textbf{ID} & \textbf{Requisito} & \textbf{Prioridade} \\
        \hline
        RG001 & Armazenar comprimidos de diferentes formatos, tamanhos e textura & \textit{Must have}\\ 
        \hline
        RG002 & Facilitar o abastecimento do estoque de medicamentos & \textit{Must have} \\
        \hline
        RG003 & Garantir um ambiente de armazenamento seguro (temperatura, umidade e sem micro-organismos) & \textit{Must have}\\ 
        \hline
        RG004 & Armazenar comprimidos de no mínimo 5 pacientes  & \textit{Should have}\\ 
        \hline
        RG005 & Emitir sinais de alerta & \textit{Should have}\\ 
        \hline
        RG006 & Assegurar que a dose medicamentosa esteja correta para um paciente específico & \textit{Must have}\\
        \hline
        RG007 & Notificar os usuários em relação a doses não tomadas & \textit{Must have}\\
        \hline
        RG008 & Regular o horário de ministração medicamentos & \textit{Must have}\\ 
        \hline
        RG009 &  Identificação e comprovar o paciente  & \textit{Should have}\\
        \hline
        RG010 & Ter interface de fácil uso no dispositivo e no aplicativo & \textit{Must have}\\
        \hline
        RG011 & Visualização do histórico de medicamentos dos pacientes & \textit{Must have}\\ 
        \hline
    \end{tabular}\label{tab:req_gerais}
\end{table}


\section{Requisitos dos Subsistemas}
\subsection{Requisitos Estruturais}

Os requisitos referentes a parte estrutural do projeto são deliberados na Tab. \ref{tab:req_estruturais}.

\begin{table}[H]
        \centering
    \caption{Requisitos Estruturais}
	\begin{adjustbox}{max width = \textwidth}
        \begin{tabular}{|L{1.5cm}|L{8.5cm}|C{2.5cm}|C{3.5cm}|}
            \hline
            \rowcolor[HTML]{A8DADC}
            \textbf{ID} & \textbf{Requisito} & \textbf{Prioridade} & \textbf{Ambiente} \\
            \hline
            RE001 & Armazenar medicamentos sólidos com isolamento admissível para temperatura, umidade e luminosidade & \textit{Must have} & Contêiner\\ 
            \hline
            RE002 & Cobrir grande variedade de dimensões dos medicamentos comerciais no sistema de seleção & \textit{Should have} & Contêiner \\
            \hline
            RE003 & Ter armazenamento removível para limpeza & \textit{Must have} & Mesa de apoio\\
            \hline
            RE004 & Verificar materiais adequados para alocação de medicamentos na máquina & \textit{Must have} & Estrutura geral\\
            \hline
            RE005 &  Garantir estrutura de proteção para componentes internos & \textit{Must have} & Carcaça\\
            \hline
            RE006 & Confirmar a saída de um único medicamento sólido do contêiner com alta acurácia & \textit{Must have}  & Comporta\\ 
            \hline
            RE007 & Ter estoque de recipientes que receberão a dose selecionada de medicamentos & \textit{Must have} & Reservatório de copos\\
            \hline
            RE008 & Garantir que os medicamentos selecionados irão cair no recipiente do paciente & \textit{Must have}  & Funil de Saída\\ 
            \hline
            RE009 & Posicionar os copos dos pacientes no local de deposição da medicação & \textit{Must have} & Atuador linear\\
            \hline
            RE010 & Deslocar o recipiente com dose selecionada de medicamentos até o local de retirada & \textit{Must have} & Esteira\\ 
            \hline
            RE011 & Designar local de retenção do recipiente com medicamentos selecionados com falha & \textit{Must have} & Zona de retorno\\
            \hline
            RE012 & Disponibilizar local de retirada dos recipientes com dosagem correta & \textit{Must have}  & Área de espera\\ 
            \hline
            RE013 & Dispor um sistema estrutural adequado para manutenções preventivas & \textit{Should have} & Estrutura geral \\
            \hline
        \end{tabular}
	\end{adjustbox}
	\label{tab:req_estruturais}
\end{table}


\subsection{Requisitos Eletrônicos}

Os requisitos referentes a parte eletrônica do projeto são deliberados na Tab. \ref{tab:req_ele_sensor1}.

\begin{table}[H]
    \centering
    \caption{Requisitos Eletrônicos}
    \label{tab:req_ele_sensor1}
	\begin{adjustbox}{max width = \textwidth}
    % \begin{adjustwidth}{-1.3cm}{}
        \begin{tabular}{|L{1.5cm}|L{8.5cm}|C{2.5cm}|C{3.5cm}|}
        \hline
        \rowcolor[HTML]{A8DADC}
        \textbf{ID} & \textbf{Requisitos} & \textbf{Prioridade} & \textbf{Ambiente} \\ \hline
        RC001 & Avisar do trancamento da porta com acesso ao estoque & \textit{Must have} & Sensoriamento\\ \hline
        RC002 & Avisar a acoplagem dos compartimentos de medicamentos & \textit{Must have} & Sensoriamento\\ \hline
        \end{tabular}
    % \end{adjustwidth}
	\end{adjustbox}
\end{table}

\begin{table}[H]
    \centering
    %\caption{Requisitos Eletrônicos}
    %\label{tab:req_ele_sensor2}
	\begin{adjustbox}{max width = \textwidth}
    % \begin{adjustwidth}{-1.3cm}{}
        \begin{tabular}{|L{1.5cm}|L{8.5cm}|C{2.5cm}|C{3.5cm}|}
        \hline
        \rowcolor[HTML]{A8DADC}
        \textbf{ID} & \textbf{Requisitos} & \textbf{Prioridade} & \textbf{Ambiente} \\ \hline
        RC003 & Detectar temperaturas fora da faixa de conservação & \textit{Must have} & Sensoriamento\\ \hline
        RC004 & Detectar umidade relativa do ar fora da faixa de conservação & \textit{Must have} & Sensoriamento\\ \hline
        RC005 & Detectar a passagem de medicamento após sair do contêiner individual &  \textit{Must have} & Sensoriamento\\ \hline
        RC006 & Assegurar a presença do copo no local de despejo do medicamento & \textit{Must have} & Sensoriamento\\ \hline
        RC007 & Detectar a passagem de medicamento no local de despejo do medicamento & \textit{Must have} & Sensoriamento\\ \hline
        RC008 & Assegurar quantidade de comprimidos no copo & \textit{Should have} & Sensoriamento\\ \hline
        RC009 & Assegurar o transporte do recipiente para a área de espera do dispensador & \textit{Must have} & Sensoriamento\\ \hline
        RC010 & Assegurar que o copo não aprovado irá para a zona de retorno de medicamentos & \textit{Should have} & Sensoriamento\\ \hline
        RC011 & Assegurar a retirada do recipiente por funcionário autorizado & \textit{Should have} & Sensoriamento\\ \hline
        RC012 & Detectar a saída do copo do dispensador & \textit{Must have} & Sensoriamento\\ \hline
        RC013 & Detectar falhas de sensores no módulo de medição & \textit{Should have} & Sensoriamento\\ \hline
        RC014 & Assegurar o abastecimento dos estoques por funcionário autorizado & \textit{Must have} & Sensoriamento\\ \hline
        RC015 & Detecção de funcionário autorizado por autenticação física & \textit{Must have} & Sensoriamento \\ \hline
        RC016 & Armazenar dados locais na central de dados & \textit{Must have} & Comunicação\\ \hline
        RC017 & Haver um sistema embarcado na central de controle & \textit{Must have} & Comunicação\\ \hline
        RC018 & Processar dados dos módulos de medição  pela central & \textit{Must have} & Comunicação\\ \hline
        RC019 & A central de controle deve funcionar de forma desconectada da internet & \textit{Must have} & Comunicação\\ \hline
        RC020 & Comunicar a central de controle  com o módulo de visualização & \textit{Must have} & Comunicação\\ \hline
        RC021 & Comunicar a central de controle com os módulos de medição & \textit{Must have} & Comunicação\\ \hline
        RC022 & Comunicar a central de controle com os atuadores & \textit{Must have} & Comunicação\\ \hline
        RC023 & Comunicar a  central de controle com a central de dados (\textit{back-end}) & \textit{Must have} & Comunicação\\ \hline
        RC024 & Assegurar a seleção correta do medicamento no estoque a partir dos dados cadastrados no \textit{back-end} & \textit{Must have} & Comunicação\\ \hline
        RC025 & Detectar falhas de comunicação no módulo de medição & \textit{Should have} & Comunicação\\ \hline
        RC026 & Controlar o transporte do recipiente & \textit{Must have} & Controle \\ \hline
        RC027 & Controlar os atuadores & \textit{Must have} & Controle \\ \hline
        RC028 & Controlar o trancamento da porta com acesso ao estoque & \textit{Must have} & Controle \\ \hline
        RC029 & Controlar o trancamento da porta de saída do copo no dispensador & \textit{Must have} & Controle \\ \hline
        RC030 & Controlar a temperatura interna & \textit{Should have} & Controle \\ \hline
        \end{tabular}
    % \end{adjustwidth}
	\end{adjustbox}
\end{table}


\subsection{Requisitos Energéticos}

Os requisitos referentes a parte energética do projeto são deliberados na Tab. \ref{tab:req_energeticos}.

\begin{table}[H]
    \centering
    \caption{Requisitos Energéticos}
    \begin{adjustbox}{max width = \textwidth}
    % \begin{adjustwidth}{-1.3cm}{}
        \begin{tabular}{|L{1.5cm}|L{8.5cm}|C{2.5cm}|C{3.5cm}|}
        \hline
        \rowcolor[HTML]{A8DADC}
        \textbf{ID} & \textbf{Requisito} & \textbf{Prioridade} & \textbf{Ambiente} \\
        \hline
        RA001 & Fornecer a energia necessária para os componentes eletrônicos e para os atuadores & \textit{Must have}  & Fonte de alimentação principal \\ 
        \hline
        RA002 & Atender os critérios de eficiência energética e segurança do produto & \textit{Must have} & Fonte de alimentação principal \\
        \hline
        RA003 & Ser provido de um sistema de backup quando houver a interrupção do fluxo de energia da fonte principal & \textit{Must have} & Fonte de alimentação secundária \\ 
        \hline
        RA004 & Garantir uma operação versátil e eficaz do movimento mecânico da estrutura dos compartimentos dos medicamentos sólidos & \textit{Must have} & Sistema eletromecânico\\ 
        \hline
        RA005 & Garantir uma operação eficaz para a abertura e fechamento das comportas & \textit{Must have} & Sistema eletromecânico \\ 
        \hline
        RA006 & Garantir a alimentação adequada para o funcionamento do atuador elétrico linear& \textit{Must have} & Fonte de alimentação principal \\ \hline
        RA007 & Acionar o mecanismo de locomoção do recipiente com a dose medicamentosa& \textit{Must have} & Sistema eletromecânico\\ 
        \hline
        \end{tabular}
    % \end{adjustwidth}
    \end{adjustbox}
    \label{tab:req_energeticos}
\end{table}

\subsection{Requisitos de Software}
Os requisitos referentes à solução em software são deliberados na Tab. \ref{tab:req_software}.

\begin{table}[H]
    \centering
    \caption{Requisitos de Software}
	\begin{adjustbox}{max width = \textwidth}
    % \begin{adjustwidth}{-1.3cm}{}
        \begin{tabular}{|L{1.5cm}|L{8.5cm}|C{2.5cm}|C{3.5cm}|}
        \hline
        \rowcolor[HTML]{A8DADC}
        \textbf{ID} & \textbf{Requisito} & \textbf{Prioridade} & \textbf{Ambiente} \\
        \hline
        RS001 & Cadastrar o administrador do sistema & \textit{Must have}& \textit{Front-end}/\textit{Spring}\\ 
        \hline
        RS002 & Cadastrar um enfermeiro e suas respectivas informações como {nome completo, sexo, idade, etc.} & \textit{Must have} & \textit{Front-end}/\textit{Spring} \\
        \hline
        RS003 & Cadastrar um paciente e suas respectivas informações como {nome completo, sexo, idade, etc.} & \textit{Must have} & \textit{Front-end/Spring} \\ 
        \hline
        RS004 & Cadastrar receita médica e suas respectivas especificações  & \textit{Must have} & \textit{Front-end/Spring} \\ 
        \hline
        RS005 & Cadastrar medicamento e suas respectivas informações, como {nome, \textit{tag}, lote, validade, etc.} & \textit{Must have} & \textit{Front-end/Spring} \\ 
        \hline
        RS006 & Capturar informações do banco de dados para criação do histórico de medicação do paciente & \textit{Must have} & \textit{Spring} \\
        \hline
        RS007 & Exibir histórico de medicação do paciente & \textit{Must have} & \textit{Front-end/Spring} \\
        \hline
        \end{tabular}
    % \end{adjustwidth}
	\end{adjustbox}
	\label{tab:req_software}
\end{table}

\begin{table}[H]
    \centering
    %\caption{Requisitos de Software}
	\begin{adjustbox}{max width = \textwidth}
    % \begin{adjustwidth}{-1.3cm}{}
        \begin{tabular}{|L{1.5cm}|L{8.5cm}|C{2.5cm}|C{3.5cm}|}
        \hline
        \rowcolor[HTML]{A8DADC}
        \textbf{ID} & \textbf{Requisito} & \textbf{Prioridade} & \textbf{Ambiente} \\
        \hline
        RS008 & Alertar o enfermeiro em relação a medicação de um paciente & \textit{Must have} & \textit{Front-end}/\textit{Back-end} \\ 
        \hline
        RS09 & Alertar o enfermeiro quando a medicação de algum dos pacientes assistidos por ele estiver próxima de esgotar & \textit{Must have} & \textit{Front-end}/\textit{Spring} \\
        \hline
        RS010 & Comunicar com o dispensador com protocolos MQTT e/ou HTTP & \textit{Must have} & \textit{Spring} \\ 
        \hline
        RS011 & Autenticar login entre perfis separados (administrador, enfermeiro(a)) & \textit{Should have} & \textit{Front-end}/\textit{Back-end} \\ 
        \hline
        RS012 & Exibir tela para que o enfermeiro confirme se o paciente recebeu a medicação ou não & \textit{Must have} & \textit{Front-end}/\textit{Back-end} \\ \hline
        RS013 & Exibir dados cadastrais dos perfis existentes & \textit{Must-have} & \textit{Front-end}/\textit{Back-end} \\ \hline
        RS014 & Atualizar dados cadastrais do perfis existentes & \textit{Must have} & \textit{Front-end}/\textit{Back-end} \\ \hline
        RS015 & Remover logicamente usuários existentes quando for necessário & \textit{Must-have} & \textit{Front-end}/\textit{Back-end} \\ \hline
        \end{tabular}
    % \end{adjustwidth}
	\end{adjustbox}
	\label{tab:req_software_2}
\end{table}
