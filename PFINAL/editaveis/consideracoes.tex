\chapter[Considerações Finais]{Considerações Finais}

Este trabalho foi desenvolvido no contexto da disciplina Projeto Integrador 2, na Universidade de Brasília (UnB) no Campus da Faculdade do Gama (FGA), e este documento consiste no último ponto de controle da matéria. 

%foi aprimorado a concepção da ideia do produto? foi melhor desenvolvida.
%grau de conhecimento técnico 
%aplicamos princípios de engenharia em uma solução idealizada e aprendemos a detectar
%as limitações dos sistemas físicos sobre desse modelo, considerando fatores
%técnicos de todas as engenharias e sua interação mútua.
%Está etapa do trabalho teve como foco a desenvolvimento dos planos de construção
%de cada subsistema, assim como, o planejamento de integração entre as áreas.

A partir desse projeto, tem-se a possibilidade de desenvolver trabalhos futuros para outros públicos alvos como hospitais, UPAs (Unidade de Pronto Atendimento) e clínicas, com a finalidade de facilitar o gerenciamento da medicação. 

Além do mais, é viável associar, com algumas alterações, esse projeto com um equipamento que faça o recorte de cada comprimido presente na cartela, substituindo, assim, a tarefa de fracionamento manual realizada pelos colaboradores. 

Algumas ideias para implementação futura são a criação, no aplicativo, de avisos do controle de estoque de medicamentos e a permissão da família do residente do lar de idosos acompanharem o histórico de medicações administradas ao paciente.


