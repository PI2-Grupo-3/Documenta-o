\makeatletter\@openrightfalse

\chapter[Introdução]{Introdução}

Segundo a Organização Mundial da Saúde (OMS), tecnologia em saúde é um conjunto de conhecimentos e habilidades sistematizados por meio de equipamentos, medicamentos, vacinas, procedimentos e sistemas organizacionais e de suporte, os quais têm como objetivo promover a saúde, melhorar a qualidade de vida, prevenir e tratar doenças e reabilitar pessoas.\cite{OMS_2010}

Dessa forma, medicamentos são produtos farmacêuticos tecnicamente obtidos ou elaborados, com finalidade profilática, paliativa, para tratamento de doenças infecciosas e para redução dos sintomas de doenças crônicas. Entretanto, medicamentos também apresentam riscos, especialmente quando são utilizados da maneira inadequada \cite{Gimenes2016}. Consequentemente, o recurso terapêutico farmacológico ainda é o mais utilizado para o tratamento de doenças, uma vez que, oferecem uma chance de sobrevida maior comparado a outros tratamentos \cite{Dal_2012}.

Além do mais, a população idosa apresenta níveis de morbidade maiores que o da população em geral e, desse modo, tem um maior consumo de medicamentos e uma demanda superior por serviços de saúde \cite{Dal_2012}. O envelhecimento é um processo comum, gradual e inelutável para todos os seres humanos, e, com o aumento da expectativa de vida ao longo dos anos, a população idosa teve um acréscimo significante. Conforme a projeção da população de 2020 feita pelo IBGE, no Brasil existem mais de 30 milhões de pessoas acima de 60 anos, o que representa 14\% da população do país e esse percentual tende a dobrar nas próximas três décadas \cite{IBGE_2020}.  

De modo consequente, com o envelhecimento populacional e o aumento da sobrevida de indivíduos com capacidade física, cognitiva e mental debilitadas, natural do processo de envelhecimento, muitas vezes leva ao extravio da independência e autonomia dos idosos \cite{Freitas_2006}, isto implica em um aumento da demanda por cuidados, apoio e supervisão de familiares. Porém, devido a muitas famílias não possuírem disponibilidade ou habilidades necessárias para tomar conta de seus idosos de maneira a atender às suas necessidades específicas, muitos idosos são inseridos em instituições de longa permanência (ILPI) ou clínicas geriátricas, as quais poderão ser as alternativas mais adequadas para atender a essa etapa da vida (\citeauthor{Ipea},\citeyear{Ipea}; \citeauthor{Silva2013},\citeyear{Silva2013}).

Sendo assim, o presente estudo busca realizar a gerência de medicamentos em residenciais sênior a partir da construção de um dispensador coletivo de medicamentos sólidos, nomeado de \textit{PillWatcher}, o qual realiza de maneira otimizada e automatizada o armazenamento e facilita o trabalho dos profissionais qualificados e responsáveis por ministrar a rotina de medicamentos da instituição por meio da separação das doses medicamentosas individuais. Outrossim, o dispositivo visa assegurar a qualidade dos medicamentos por meio de condições adequadas de armazenamento e de um controle de estoque eficaz.
