\chapter[Avaliação Interna]{Avaliação Interna}


% Lembrar de transferir isso para os apêndices
\begin{table}[H]
    \centering
    \begin{adjustbox}{max width = \textwidth}
    % \begin{adjustwidth}{-1.9cm}{}
        \begin{tabular}{|G{4cm}|c|L{10cm}|}
        \hline
        \rowcolor[HTML]{E63946}
        \multicolumn{3}{|c|}{\textbf{\color{white}Projeto PillWatcher}}                                              \\ \hline
        \rowcolor[HTML]{1D3557}\multicolumn{3}{|c|}{\textbf{\color{white}Ponto de Controle 1}} \\ \hline
        \rowcolor[HTML]{457B9D}\multicolumn{1}{|c|}{\color{white}\textbf{Nome}} &
          \multicolumn{1}{c|}{\color{white}\textbf{Matrícula}} &
          \multicolumn{1}{c|}{\textbf{\color{white}Descrição da Contribuição}} \\ \hline
        \rowcolor[HTML]{A8DADC}\multicolumn{3}{|c|}{\textbf{Grupo Técnico de Estrutura (Engenharia Aeroespacial/Automotiva)}} \\ \hline
        Diogo P. Sousa & 12/0115590  &  Auxílio no levantamento dos medicamentos mais utilizados no abrigo de idosos, para que se pudesse ter um parâmetro do numero mínimo de contêineres que atenderiam a solução apresentada. Auxílio no levantamento das propriedades dos materiais a serem utilizados no projeto.    \\ \hline
        Fabrício de A. Oliveira & 16/0027772 &  Análise e proposição de soluções para a problemática apresentada. Participação na definição do escopo mecânico-estrutural. Pesquisa de alguns possíveis materiais. Projeto CAD do contêiner individual de medicamentos, seus componentes internos e sua mesa de apoio. Pesquisa de parte dos custos de prototipagem do projeto. Análise e proposição conjunta dos possíveis riscos estruturais e mecânicos.    \\ \hline
        Luso de J. Torres & 15/0051808 &  Auxílio na Elaboração do TAP; Elaboração conjunta da EAP estrutural; Estudo de soluções gerais aplicáveis ao projeto; Refino dos objetivos gerais e específicos do projeto; Auxílio no levantamento de pontos no questionário com o farmacêutico; Elaboração conjunta dos requisitos estruturais; Validação dos requisitos técnicos de outras áreas; organização de entregáveis na área de estruturas e acompanhamento geral dos entregáveis; Elaboração conjunta do cronograma de estruturas; Elaboração do fluxograma de operação da solução estrutural; Auxílio na validação da documentação técnica do projeto; análise de conformidade da solução entre as áreas; Pesquisa conjunta dos componentes de prototipagem do projeto; Elaboração conjunta dos riscos estruturais.  \\ \hline
        Marcos Paulo R. Garcia & 16/0014123 &  Estudo de possíveis soluç{\~o}es para solucionar os requisitos de projeto. Construção conjunta do projeto CAD dos componentes estruturais, com seu devido empacotamento e métricas iniciais. Pesquisa de materiais utilizáveis nos componentes da estrutura. Auxílio na geometria de disposição dos compartimentos, acoplamento de fusos com engrenagens e motores de passo, pontos de ancoragem na estrutura tubular. Auxílio na construção conjunta dos requisitos estruturais do projeto. Auxílio na construção conjunta dos custos estruturais do projeto. Auxílio na construção conjunta dos possíveis riscos estruturais e gerais do projeto. Análise inicial de cotagens mínimas de sustentaç{\~a}o e resist{\^e}ncia iniciais da estrutura tubular e de seus subsistemas.    \\ \hline
        \rowcolor[HTML]{A8DADC}\multicolumn{3}{|c|}{\textbf{Grupo Técnico de Controle e Alimentação (Engenharia Energia/Eletrônica)}} \\ \hline
        Gabriel G. Carmona & 16/0028558 &   Auxílio na definição dos requisitos gerais do cliente por meio de uma reunião com o farmacêutico; Auxilio na definição de requisitos técnicos de eletrônica; Auxílio na solução do escopo de eletrônica; Auxílio nos fluxogramas de arquitetura de eletrônica e o de integração;    \\ \hline
        Luiza Carolina C. Gonçalves & 13/0143791 &  Auxílio na elaboração do EAP de energia; auxílio na elaboração do TAP; Estudo de possíveis soluções para solucionar os requisitos do projeto; Colaboração na elaboração dos requisitos  técnicos de energia; Pesquisas de motores utilizáveis no projeto; Auxílio na solução do escopo de energia; Auxílio na construção conjunta dos custos e possíveis riscos energéticos e gerais do projeto;     \\ \hline
        Rebeka P. Gomes &   16/0017491 & Auxílio na elaboração do EAP e TAP de energia; Estudo de possíveis soluções para solucionar os requisitos do projeto; Auxílio na solução do escopo de energia; Auxílio na definição de requisitos técnicos de energia; Auxílio nos fluxogramas de energia e no de integração; Auxílio na realização da introdução; Auxílio na construção dos custos e riscos energéticos;  \\ \hline
        Sofia C. Fontes & 16/0018234 &    Realização da TAP; realização EAP; realização da introdução, problematização, justificativa e objetivos; definição dos requisitos gerais do cliente por meio de uma reunião com o farmacêutico e questionário enviado para instituições; Auxilio na definição de requisitos técnicos de eletrônica e validação dos requisitos técnicos das demais áreas de engenharia para garantir que a arquitetura da solução atenda às necessidades do cliente; auxílio na solução do escopo de eletrônica e análise da solução das outras áreas de desenvolvimento; organização e acompanhamento dos entregáveis de toda equipe; realização dos fluxogramas de arquitetura de eletrônica e o de integração e validação da documentação técnica do projeto.  \\ \hline
        Tiago R. Pereira & 16/0072620 &   Realização da EAP eletrônica; Definição dos requisitos gerais do cliente por meio de uma reunião com o farmacêutico; Auxilio na definição de requisitos técnicos de eletrônica e validação dos requisitos técnicos das demais áreas de engenharia; Auxílio na solução do escopo de eletrônica; Auxílio nos fluxogramas de arquitetura de eletrônica e o de integração; Formatação das tabelas, figuras e detalhes específicos do documento geral; Organização e acompanhamento dos entregáveis da equipe Controle e Alimentação;  \\ \hline
        \rowcolor[HTML]{A8DADC}\multicolumn{3}{|c|}{\textbf{Grupo Técnico de Software}} \\ \hline
        Amanda V. Pires & 15/0004796 &  Para esse primeiro ponto controle, me disponibilizei sempre para ajudar o grupo. Participei da definição dos requisitos, planejamento da solução de software, definição da arquitetura, integração com a máquina (Internet das coisas), desenvolvimento da EAP e do planejamento das sprints. Considero que fui bem participativa e ajudei bastante com as documentações e definições a serem feitas.\\ \hline
        Filipe D. Lima & 16/0006163 &  Auxílio na elicitação de requisitos funcionais, refatoração do escopo do projeto. Diretor técnico responsável pelo levantamento da arquitetura da solução e coordenação da equipe como um todo. Auxílio no desenvolvimento de documentos e organização da documentação/wiki do projeto. \\ \hline
        Gabriela C. de Moraes &  16/0006872 &  Documentação da metodologia a ser utilizada, auxílio no levantamento e escrita dos requisitos, contribuição na escrita da solução de software, na Lista É/Não É e nos custos de software.  \\ \hline
        Geovanne S. Saraiva &  15/0035756 &  Gerenciamento dos riscos, dojo de Ágil, fluxograma eletrônica-software, levantamento de requisitos, organização do cronograma, reuniões com o Chaim para validar ideias \\ \hline
        Kamilla C. Souza &  16/0010969 &  Auxílio na refatoração do escopo do projeto, levantamento dos requisitos funcionais da aplicação, documentação do método utilizado para elicitação dos requisitos, desenvolvimento dos Rich Pictures relacionados a solução de software e interação entre os componentes, levantamentos dos riscos técnicos e documentação da entrevista realizada com o cliente. \\ \hline
        \end{tabular}
    % \end{adjustwidth}
    \end{adjustbox}
\end{table}