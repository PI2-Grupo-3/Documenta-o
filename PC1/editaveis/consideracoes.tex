\chapter[Considerações Finais]{Considerações Finais}

Este trabalho foi desenvolvido no contexto da disciplina Projeto Integrador 2,
pela Universidade de Brasília UnB no Campus da Faculdade do Gama (FGA), e consiste no
primeiro de três pontos de controle da matéria. Dessa forma, para o primeiro ponto de controle foram necessárias muitas pesquisas sobre o processo de armazenamento e administração de medicamentos, protocolos farmacêuticos e normas técnicas de tecnologias em saúde. 

Dessa forma, a partir do dimensionamento preliminar das Áreas de estrutura, controle, automação e software, foi estabelecida uma solução para o problema proposto. Após repetidos esforços pelo consentimento dos professores da abordagem proposta, refinou-se a concepção final da solução.  

Após a elaboração do cronograma, dos custos e do escopo de cada engenharia, foi possível realizar o estudo de viabilidade. Percebe-se que o projeto pode ser elaborando nos meses de agosto a novembro seguindo o cronograma de \textit{sprints}. Além do mais, por se tratar de um semestre remoto não existirá a possibilidade da construção de um protótipo final e assim, os custos de projeto serão utilizados para documentação.



