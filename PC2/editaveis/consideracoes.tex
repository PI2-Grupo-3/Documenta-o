\chapter[Considerações Finais]{Considerações Finais}

Este trabalho foi desenvolvido no contexto da disciplina Projeto Integrador 2,
pela Universidade de Brasília UnB no Campus da Faculdade do Gama (FGA), e consiste no
segundo de três pontos de controle da matéria. 

%foi aprimorado a concepção da ideia do produto? foi melhor desenvolvida.
%grau de conhecimento técnico 
%aplicamos principios de engenharia em uma solução idealizada e aprendemos a detectar
%as limitações dos sistemas físicos sobre desse modelo, considerando fatores
%técnicos de todas as engenharias e sua interação mútua.
%Está etapa do trabalho teve como foco a desenvolvimento dos planos de construção
%de cada subsistema, assim como, o planejamento de integração entre as áreas.

No decorrer do segundo ponto de controle, a concepção da ideia do produto passou por algumas modificações. Esse desenvolvimento pode ser justificado em termos de aspectos técnicos associados à solução. Basicamente, ao aplicar princípios de engenharia em uma solução idealizada, detectou-se as limitações dos sistemas físicos sobre esse modelo, considerando soluções disponíveis no mercado e a interação mútua entre todas as engenharias para as decisões de projeto.

Indo a fundo no aspecto de decisões, foram realizadas mudanças na área de \textit{software}, onde foi necessário postergar a construção do \textit{front-end}. Em um aspecto geral, corrigiu-se tópicos exigidos por parte dos docentes no seu \textit{feedback} sobre o Ponto de Controle 1 para todas as engenharias.

Do ponto de vista financeiro, poucas alterações foram implementadas. Entretanto, o valor final do projeto aumentou em razão da delimitação técnica de certos dispositivos, 
redimensionamento de componentes eletrônicos e elétricos e refinamento da solução estrutural.

