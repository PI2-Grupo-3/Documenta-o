\chapter[Gerenciamento]{Gerenciamento}

O presente capítulo apresenta e descreve as metodologias que serão utilizadas na gerência do projeto. Assim como, expõe os papéis e atividades que serão realizadas ao longo do projeto.

Para gerenciamento do projeto, serão utilizados artefatos relacionados as áreas do conhecimento sugeridas pelo PMBOK, além das metodologias ágeis Scrum e Kanban.

\section{Metodologia}

\subsection{Scrum}
Scrum é um \textit{framework} simples que auxilia a equipe a resolver problemas complexos enquanto entregam, produtivamente e criativamente, produtos de maior valor possível, e que visa a colaboração efetiva do time. \cite{SCRUM.ORG_2020}

Sendo assim, os eventos, artefatos e papéis sugeridos, explicados abaixo, serão os principais guias para o gerenciamento do presente projeto.

\subsubsection*{1. Eventos:}

\begin{itemize}
    \item \textbf{Sprint}
    
    É o ponto-chave do \textit{Scrum}. Consiste de um período definido pelo time no qual será criado um incremento do produto que seja utilizável e que apresente potencial para ser lançado. \cite{THESCRUMGUIDE_2018}
    
    Serão \textit{sprints} semanais iniciando na Segunda-feira.
    
    \item \textbf{Planejamento e Revisão da Sprint}
    
    O planejamento da \textit{sprint} é o evento que ocorre no início dessa e cujo objetivo é planejar e organizar todo o trabalho que será feito durante esse intervalo.
    A revisão é feita no final do ciclo e tem como finalidade analisar e avaliar o resultado, além de sinalizar melhorias para as próximas iterações.

    Ambos eventos serão realizados na reunião geral que ocorre às 20:00h nas segundas-feiras.
    
    \newpage
    
    \item \textbf{Daily Meeting}
    
    Evento de 15 minutos realizados todos os dias  durante a \textit{sprint} com horário inicial fixo \cite{THESCRUMGUIDE_2018}. Esse evento visa acompanhar a realização das atividades e verificar se há algum empecilho ao cumprimento dessas e caso seja detectado algo, deve-se mitigar o mais rápido possível.

    Realizada todos os dias pelo \textit{Telegram} às 20:00h.
\end{itemize}

\subsubsection*{2. Artefatos:}
\begin{itemize}
    \item \textbf{Product e Sprint Backlog}
    
    O \textit{Product Backlog} é o artefato que armazena todo o escopo do produto a ser entregue no final do projeto. O \textit{Sprint Backlog} é composto pelas atividades que serão realizadas durante uma \textit{sprint}.

    Os dois estarão presentes no \textit{Kanban} que será introduzido posteriormente.
\end{itemize}

\subsubsection*{3. Papéis:}
\begin{itemize}
    \item \textbf{Scrum Master}
    
    É o responsável por assegurar que a equipe respeite e siga os valores e as práticas do \textit{Scrum}, por garantir que a equipe não se comprometa excessivamente com relação àquilo que é capaz de realizar durante uma \textit{sprint}, além de atuar como facilitador, removendo possíveis obstáculos apontados pela equipe \cite{DESENVOLVIMENTOAGIL_2013}. O coordenador geral será responsável por exercer essa função.
    
    \item\textbf{Product Owner}
    
    O \textit{Product Owner} é responsável por maximizar o valor do produto resultante do trabalho do time de desenvolvimento \cite{THESCRUMGUIDE_2018}. Esse papel será desempenhado pelo diretor de qualidade.
    
    \item \textbf{Time de desenvolvimento}
    
    Consiste de pessoas que serão responsáveis pela entrega do incremento potencialmente lançável a cada final de \textit{sprint} \cite{THESCRUMGUIDE_2018}. No projeto, os alunos dos sub-grupos de estrutura, software controle e alimentação agem como times de desenvolvimentos diferentes.
\end{itemize}


\subsection{Kanban}

Método para acompanhamento e organização de atividades. O uso desse quadro, possibilita que o time consiga visualizar suas responsabilidades e o progresso de todas atividades daquela \textit{sprint} e do projeto. A ferramenta a ser utilizada para esse propósito é o \textit{Notion} e foi estabelecido o seguinte fluxo: \textit{product backlog}, \textit{sprint backlog}, a fazer, fazendo, feito.

\subsection{PMBOK}

O PMBOK consiste em um conjunto de melhores práticas que visam padronizar e auxiliar no gerenciamento de projetos possibilitando maiores chances de sucesso. Consequentemente, o projeto utiliza essa metodologia para realizar os entregáveis de documentação. Os artefatos confeccionados que englobam as áreas de conhecimento sugeridas por esse:


\begin{enumerate}
    \item[ ]
    \begin{itemize}
        \item[ ]
        \begin{itemize}
            \item TAP - Termo de Abertura de Projeto
            \item EAP - Estrutura Analítica de Projeto
            \item Plano de Gerenciamento de Tempo (Cronograma de Atividades)
            \item Plano de Gerenciamento Recursos Humanos
            \item Plano de Gerenciamento de Custos e Aquisições
            \item Plano de Gerenciamento de Comunicação
            \item Plano de Gerenciamento de Riscos
        \end{itemize}
    \end{itemize}
\end{enumerate}


\section{Termo de Abertura de Projeto (TAP)}

O documento do termo de Abertura de Projeto encontra-se no Apêndice \ref{ATP_app}

\section{Estrutura Analítica de projeto (EAP)}

A EAP é uma subdivisão hierárquica do trabalho do projeto em partes menores, mais facilmente gerenciáveis. Seu objetivo primário é organizar o que deve ser feito para produzir as entregas do projeto \cite{EUAX_2018}. 

Para facilitar a compreensão do projeto, na EAP foi utilizada a estratégia híbrida: por entregas (pontos de controle) e por subprojeto. Assim, a EAP encontra-se no Apêndice \ref{EAP_app}.


\section{Gerenciamento de Recursos Humanos}

O plano de Recursos Humanos encontra-se no apêndice \ref{RH_app}.

\subsection{Organização da Equipe do Projeto}
A equipe é composta por 14 alunos, divididos em sub-equipe de software, sub-equipe de estrutura e sub-equipe de alimentação e controle. Sendo assim, a tabela \ref{tab:equipe} detalha os membros da equipe, a engenharia cursada e suas funções, seguindo a metodologia PMBOK.

\begin{table}[H]
\centering
\caption{Equipe do Projeto e Funções}
\label{tab:equipe}
\begin{tabular}{@{}|c|c|c|@{}}
\hline
\rowcolor[HTML]{A8DADC}
\textbf{Membro} & \textbf{Engenharia} & \textbf{Função} \\ \hline
Luso de Jesus Torres& Aeroespacial & Coordenador Geral \\ \hline
Sofia Consolmagno Fontes & Eletrônica & Diretor de Qualidade \\ \hline
Filipe Dias Soares Lima & Software & Diretor técnico \\ \hline
Tiago Rodrigues Pereira & Eletrônica & Diretor técnico \\ \hline
Marcos Paulo Ribeiro Garcia & Automotiva & Diretor técnico \\ \hline
Amanda Vieira Pires & Software & Desenvolvedor \\ \hline
Diogo Pontes Sousa & Aeroespacial & Desenvolvedor \\ \hline
Fabrício de Almeida Oliveira & Automotiva & Desenvolvedor \\ \hline
Gabriel Genari Carmona & Eletrônica & Desenvolvedor \\ \hline
Gabriela Chaves de Moraes & Software & Desenvolvedor \\ \hline
Geovanne Santos Saraiva & Software & Desenvolvedor \\ \hline
Kamilla Costa Souza & Software & Desenvolvedor \\ \hline
Luiza Carolina Caetano Gonçalves & Energia & Desenvolvedor \\ \hline
Rebeka Passos Gomes & Energia & Desenvolvedor \\ \hline
\end{tabular}
\end{table}

\subsection{Ferramentas de Comunicação e Gerenciamento de Atividades}

As ferramentas e a descrição da comunicação entre a equipe e as subequipes se encontram detalhados no Plano de Comunicação no apêndice \ref{Comunicação_app}

\section{Cronograma}
A partir da escolha da metodologia \textit{Scrum}, foi elaborado um \textit{roadmap}, o qual incorpora todas as \textit{sprints} por pontos de controle e está descrito no apêndice \ref{roadmap}.


