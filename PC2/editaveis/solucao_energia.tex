\chapter[Solução de Energia]{Solução de Energia}
\label{Solução_energia}

% Cada frente deve adc os subtópicos que acharem pertinentes

O escopo da solução energética do projeto envolve a implementação de uma fonte de alimentação principal, uma fonte de alimentação alternativa e um sistema eletromecânico. Todos esses componentes serão ocultados do campo de visão do usuário. Os objetivos da solução envolvem alcançar um custo reduzido, alta eficiência, robustez e compactação da estrutura. A configuração da solução da alimentação do sistema está representada no diagrama de blocos simplificado da figura \ref{fig:energia_alimentacao}.

\begin{figure}[H]
    \centering
    \includegraphics[width=1\textwidth]{figuras/Alimentação.png}
    \caption{Esquema do sistema de alimentação proposto.}
    \label{fig:energia_alimentacao}
\end{figure}

A carga do projeto será atendida de duas maneiras, considerando dois modos de operação: pela fonte de alimentação, quando houver energia elétrica na rede, e por uma fonte de alimentação alternativa, cuja função será atuar como uma fonte auxiliar/secundária que assegurará a ininterruptibilidade no provimento de energia em casos de falha da energia proveniente da rede primária.

\section{Sistema eletromecânico}

A escolha dos motores é baseada nas características técnicas das aplicações, nas exigências das cargas e na alimentação do sistema no que se refere ao ponto de vista mecânico para avaliar o conjugado de aceleração, de partida e o nominal \cite{santos_2016}. Essas características serão comparadas com as características dos motores para a seleção adequada dos mesmos. Os seguintes itens foram considerados:

\begin{enumerate}
    \item[ ]
    \begin{itemize}
        \item[ ]
        \begin{itemize}
            \item Redução de custos;
            \item Conjugados solicitados;
            \item Dispensabilidade ou não de regulação de velocidade;
            \item Menor espaço ocupado (horizontal e vertical);
            \item Rendimento;
           % \item Aumento de temperatura;
            \item Funcionalidade e viabilidade (benefício pelo ônus);
            \item Segurança;
          %  \item Menor exigência de potência (economia de energia).
          %  \item Menor índice de falhas;
          %  \item Facilidade de manutenção;
            \item Versatilidade do controle dinâmico.
        \end{itemize}
    \end{itemize}
\end{enumerate}

O dimensionamento e a seleção adequados de um motor para um equipamento são essenciais para garantir a confiabilidade, o desempenho e o custo do equipamento. Logo, os motores foram selecionados com base nas especificações requeridas. Após essa seleção, foi feito uma determinação final do motor, confirmando que as especificações dos motores selecionados satisfazem todos os requisitos.

%O primeiro passo para dimensionar um motor elétrico é determinar certas características do projeto, como o mecanismo de acionamento, dimensões aproximadas, distâncias percorridas e período de posicionamento.

%Junto com o tipo de mecanismo de acionamento, também será necessário determinar as dimensões, massa e coeficiente de atrito que são necessários para o cálculo de carga:

%\begin{enumerate}
 %   \item[ ]
%    \begin{itemize}
    %    \item[ ]
   %     \begin{itemize}
  %          \item Dimensões e massa (ou densidade) da carga;
 %           \item Dimensões e massa (ou densidade) de cada parte;
%            \item Coeficiente de atrito da superfície deslizante de cada peça %móvel;
  %      \end{itemize}
 %   \end{itemize}
%\end{enumerate}

%Em seguida, foi preciso designar as especificações necessárias para o equipamento:

%\begin{enumerate}
%    \item[ ]
 %   \begin{itemize}
  %      \item[ ]
   %     \begin{itemize}
    %        \item Velocidade operacional e tempo operacional;
     %       \item Distância de posicionamento e tempo de posicionamento;
      %      \item Precisão de parada;
       %%%%% \item Ambiente operacional. %(Caixa fechada, Temperatura mínima X, Temperatura Máxima Y).
      %  \end{itemize}
    %\end{itemize}
%\end{enumerate}


%Após todas as definições acima, é fundamental determinar o desempenho do motor, calculando o movimento de inércia, o torque e velocidade no eixo de acionamento.

%MOVIMENTO DE INÉRCIA: FÓRMULA
%TORQUE : FÓRMULA
%VELOCIDADE: FÓRMULA

\subsection{Motor de passo}\label{energ:motor_passo}

 O dispositivo eletromecânico responsável pelo deslocamento e disposição dos compartimentos que armazenam os medicamentos será um motor de passo de corrente contínua. Este é capaz de suprir o torque requerido pelo fuso que irá girar os compartimentos. A escolha desse motor se deu pela sua capacidade de produzir alto torque em baixa velocidade e, ao mesmo tempo, minimizar a vibração. Seu dimensionamento levou em consideração os seguintes itens: 

\begin{itemize}
    \item A velocidade máxima operacional = 158 rpm;
    \item Angulo de passo necessário = 1,80 $^\circ$deg;
    \item Velocidade angular = 1rps.
\end{itemize}

 \begin{figure}[H]
\centering
    \includegraphics[width=0.35\textwidth]{figuras/Energia_passo.PNG}
    \caption{Motor de passo selecionado para a movimentação dos compartimentos.}
    \label{fig:energia_passo}
\end{figure}

 O motor dimensionado foi o motor de passo padrão NEMA 17 - KTC-42HS40-0404, da marca Kalatec Automação, representado na figura \ref{fig:energia_passo}. Este apresenta um torque nominal 0,16 Nm com potência de 4,8 W. O motor apresenta um controle preciso de posição, torque e velocidade, estes que são necessários para fazer girar os contêineres que distribuirá os comprimidos.
 
 \subsection{Motor DC}\label{energ:motor_dc}

O dispensador contará também com uma esteira de locomoção, que tem por objetivo efetuar a locomoção dos copos com os medicamentos até a saída do produto, esta será acionada por meio de um motor DC de eixo duplo. A escolha desse dispositivo se deu ao considerar um menor custo, economia de energia e de fácil operação. Seu dimensionamento levou em consideração os seguintes itens:

\begin{itemize}
    \item Velocidade nominal = 12 rpm;
    \item Distância que o copo irá percorrer até a saída = 570 mm;
     \item Massa do conjunto = 3kg; 
\end{itemize}

\begin{figure}[H]
\centering
    \includegraphics[width=0.35\textwidth]{figuras/Energia_dc.PNG}
    \caption{Motor DC selecionado para o acionamento da esteira.}
    \label{fig:energia_dc}
\end{figure}

O motor dimensionado foi BRINGSMART 12V, da marca Hugwit Company, representado na figura \ref{fig:energia_dc}. Este apresenta um torque de 1,8 Nm com potência de 19,2 W. 
 
\subsection{Mini Atuador elétrico linear}\label{energ:atuador_linear}

O controle de abertura do reservatório de copos será por meio de um mini atuador elétrico linear. O atuador elétrico linear converte movimento rotativo em uma voltagem contínua em movimento linear, a escolha deste dispositivo de acionamento se deu pela facilidade de instalação, operação e velocidade constante. Seu dimensionamento levou em consideração os seguintes itens:

\begin{itemize}
   % \item Força linear necessária para mover o copo = 0,1 N;
    \item Distância linear que o copo precisar percorrer até a esteira = 90 mm;
    \item O tempo requerido para movimentar o copo = 5 s;
\end{itemize}

\begin{figure}[H]
\centering
    \includegraphics[width=0.5\textwidth]{figuras/Energia_atuador.PNG}
    \caption{Mini atuador elétrico selecionado para movimentar o copo até a esteira.}
    \label{fig:energia_atuador}
\end{figure}


Dessa forma, o atuador dimensionado foi o mini atuador elétrico linear representado na figura \ref{fig:energia_atuador}. Este apresenta torque de 20N, com haste de 100mm e velocidade constante de 15mm/s.

\subsection{Solenoide}\label{energ:solenoide}

O dispositivo eletromecânico responsável pela abertura e fechamento das comportas será um eletroímã solenoide elétrico - atuador \textit{Push Pull}, quando acionado ele irá abrir uma pequena porta, onde só irá descer um comprimido por vez. A escolha desse dispositivo se deu pela facilidade de instalação, operação e movimento linear.
Para a porta de saída do dispositivo, o acionamento se dará por meio de uma fechadura elétrica solenoide que a sustentará na posição aberta até que o copo com o medicamento seja entregue ao usuário.

\begin{figure}[H]
\centering
\subfloat[][Solenoide selecionada para abertura da porta dos compartimentos.]{
\includegraphics[width=0.4\textwidth]{figuras/Energia_push.PNG}
\label{fig:energia_push}}
\qquad
\subfloat[][Solenoide selecionada para o controle da porta de acesso.]{
\includegraphics[width=0.3\textwidth]{figuras/Energia_porta.PNG}
\label{fig:energia_porta}}
\caption{Solenoides selecionadas para o projeto.}
%\label{fig:globfig}
\end{figure}

A solenoide selecionada para abertura da porta dos compartimentos é da marca ECSZINX, modelo \textit{Push Pull} 12V representado na figura \ref{fig:energia_push} e para o controle da porta de acesso, é uma Mini Trava Elétrica Solenóide 12V da marca AT representado na figura \ref{fig:energia_porta}.

\section{Fonte de Alimentação Principal} \label{section:energia_fonte}

% Para o fornecimento de energia elétrica necessária ao motor e circuitos eletrônicos do dispensador, 
 
 O dispositivo que converte a energia elétrica disponibilizada pela concessionária para uma tensão, corrente e frequência exigida por algum eletroeletrônico é a fonte de alimentação. Desse modo, a fim de atender as necessidades energéticas do projeto, o dispensador Pill Watcher contará com uma fonte de alimentação comutada ou chaveada \textit{(switched-mode power supply} - SMPS), que terá prioridade no sistema de energização do equipamento. Os fatos que motivaram a escolha de desenvolver uma fonte chaveada em relação a uma fonte linear são \cite{Projeto_fonte}:
 
 %MUDAR PARA UMA TABELA COMPARATIVA
 \begin{enumerate}
    \item[ ]
    \begin{itemize}
        \item[ ]
        \begin{itemize}
            \item Menor tamanho;
            \item Menor peso;
            \item Maior eficiência;
            \item Menor geração de calor;
            \item Baixo consumo;
            \item Equipamento mais compacto;
        \end{itemize}
    \end{itemize}
\end{enumerate}

O dimensionamento das duas fontes de alimentação será determinado principalmente pelo sistema embarcado e motores que atendam ao sistema de operação, e levará em conta sua distribuição espacial no equipamento. A tabela \ref{fig:energia_carga} contém o levantamento da carga do sistema.

%A configuração contará com um sistema de proteção contra curto-circuito, sobrecarga, sobretensão, baixa tensão e ligação inversa da bateria.
 
\begin{table}[htb]
    \centering
    \caption{Levantamento da carga do projeto.}
    \label{fig:energia_carga}
    \begin{adjustbox}{max width = \textwidth}
        \begin{tabular}{|L{5cm}|C{2cm}|C{2cm}|C{2cm}|C{2cm}|C{2cm}|}
            \hline
            \rowcolor[HTML]{A8DADC}
            \textbf{Equipamento} & \textbf{Quant. (unid.)} & \textbf{Tensão (unid.) (V)} & \textbf{Corrente (unid.) (A)} & \textbf{Potência (total) (W)} \\ \hline
            Sensor de Temperatura e Umidade & 1	 & 3,3 & 0,0005 & 0,0015
            \\ \hline
           %   Sensor Fotoelétrico Emissor & 30	 & 3,3 & 0,1 & 9,9
         %   \\ \hline
              Sensor Fotoelétrico & 30	 & 5  & 0,025 & 3,75
             \\ \hline
             Sensor de Biometria & 1 & 5 & 0,12 & 0,6
             \\ \hline
             Sensor de Leitor RFID & 1	 & 5 & 0,025 & 0,125
             \\ \hline
              Raspberry Pi 4 & 1 & 5 & 3 & 15
             \\ \hline
               Microcontrolador & 4 & 5 & 0,025 & 0,5
             \\ \hline
               Display & 1 & 5 & 0,02 & 0,1
             \\ \hline
               Câmera & 1 & 3,3 & 0,2 & 0,66
             \\ \hline
            %  Teclado & 1 & X & X & X
            % \\ \hline
            %  Cls Mux/Demux & 12 & 5 & 0,000001 & 
            % \\ \hline
              Fechadura Elétrica Solenoide & 2 & 12 & 0,5 & 12
             \\ \hline
              Motor de Passo & 5 & 12 & 0,4 & 24
             \\ \hline
              Eletroímã Solenoide elétrico - atuador Push Pull & 25 & 5 & 1,12 & 140
             \\ \hline
                Motor DC & 1 & 12 & 1,6 & 19,2
             \\ \hline
                 Atuador Linear Elétrico & 1 & 12 & 2 & 24
             \\ \hline
               Driver para Motor de Passo & 5 & 12 & 0,5 & 30
             \\ \hline
                Driver IRF 520 & 28 & 12 & 2 & 672
             \\ \hline
                  Driver Ponte H & 1 & 12 & 1,6 & 19,2
             \\ \hline
             \rowcolor[HTML]{F1FAEE}
             \multicolumn{4}{|l|}{\cellcolor[HTML]{F1FAEE}Total} & 1157,14 \\
             \hline
        \end{tabular}
    \end{adjustbox}
\end{table}

A potência instalada do equipamento é igual a 1157,14 W. A fonte de alimentação e a fonte auxiliar deverão ser dimensionadas com base na corrente total e no potencial de falha do pior caso. Portanto, serão considerados os seguintes cenários representados na tabela \ref{fig:energia_cenarios}. 

Para o cenário 1, todas as cargas são acionadas ao mesmo tempo, o que é improvável de acontecer, pois a premissa do funcionamento do equipamento é que só um medicamento pode ser expelido por vez. No cenário 2, é considerado que seriam acionados ao mesmo tempo um motor de passo, uma solenoide, o atuador e as travas elétricas. 

\begin{table}[H]
    \centering
    \caption{Cenários considerados para o dimensionamento dos sistemas de alimentação.}
    \label{fig:energia_cenarios}
    \begin{adjustbox}{max width = \textwidth}
        \begin{tabular}{|L{5cm}|C{2cm}|C{2cm}|C{2cm}|C{2cm}|C{2cm}|}
            \hline
            \rowcolor[HTML]{A8DADC}
            \textbf{Equipamento} & \textbf{Cenário 1} & \textbf{Cenário 2} & \textbf{Cenário 3} & \textbf{Cenário 4} \\ \hline
            Sensor de Temperatura e Umidade & x1 & x1 & x1 & x1
            \\ \hline
           %   Sensor Fotoelétrico Emissor & 30	 & 3,3 & 0,1 & 9,9
         %   \\ \hline
              Sensor Fotoelétrico & x30	 & x30 & x30 & x30
             \\ \hline
             Sensor de Biometria &  x1	 & x1 & x1 & x1
             \\ \hline
             Sensor de Leitor RFID & x1	 & x1  & x1 & x1
             \\ \hline
              Raspberry Pi 4 & x1 & x1 & x1 & x1
             \\ \hline
               Microcontrolador & x4 & x4 & x3 & x4
             \\ \hline
               Display & x1	 & x1 & x1 & x1
             \\ \hline
               Câmera & x1 & x1 & x1 & x1
             \\ \hline
            %  Teclado & 1 & X & X & X
            % \\ \hline
            %  Cls Mux/Demux & 12 & 5 & 0,000001 & 
            % \\ \hline
              Fechadura Elétrica & x2	 & x2 & - & -
             \\ \hline
              Motor de Passo & x5 & x1 & x1 & x1
             \\ \hline
             Atuador Push Pull & x25	 & x1 & - & -
             \\ \hline
                Motor DC & x1 & x1 & x1 & -
             \\ \hline
                 Atuador Linear Elétrico & x1 & x1 & - & x1
             \\ \hline
               Driver para Motor de Passo & x5 & x1 & x1 & x1
             \\ \hline
                Driver IRF 520 & x28 & x1 & - & x1
             \\ \hline
                  Driver Ponte H & x1 & x1 & x1 & -
             \\ \hline
             \rowcolor[HTML]{F1FAEE}
             \multicolumn{1}{|l|}{\cellcolor[HTML]{F1FAEE}Corrente Total (A) } & 98,98 & 13,82 & 8,29 & 9,12 \\
             \hline
        \end{tabular}
    \end{adjustbox}
\end{table}

Para os cenários 3 e 4 são considerados as seguintes premissas: 
    
    \begin{itemize}
        
        \item O medicamento não pode ser liberado da comporta enquanto o contêiner estiver em movimento, ou seja, o motor de passo e a solenoide não podem ser acionados ao mesmo tempo;
        
        \item Não será possível colocar um copo na esteira em movimento, ou seja, o atuador elétrico e o motor DC não podem ser acionados ao mesmo tempo;
        
        \item Não será possível liberar um medicamento enquanto a esteira estiver em movimento, ou seja, a solenoide, o motor DC e o atuador elétrico não podem ser acionados ao mesmo tempo;
        
        \item O contêiner pode estar em movimento enquanto um copo com medicamento está sendo levado para a saída do equipamento, ou seja, o motor de passo e o motor DC podem ser acionados ao mesmo tempo;
        
        \item O contêiner pode estar em movimento enquanto um copo é colocado na esteira, ou seja, o motor de passo e o atuador elétrico podem ser acionados ao mesmo tempo;
        
        \item O contêiner pode estar em movimento enquanto a porta de saída do remédio é aberta, ou seja o motor de passo e a trava elétrica podem ser acionados ao mesmo tempo;

    \end{itemize}

Com base nas premissas apresentadas acima, considerou-se no cenário 3 que um motor de passo e o motor DC estariam funcionando ao mesmo tempo. Já no cenário 4, é considerado que um motor de passo e o atuador elétrico funcionariam ao mesmo tempo.

Ao avaliar os cenários propostos, percebe-se que, para o dimensionamento dos sistemas de alimentação, os cenários 3 e 4 são os mais apropriados, pois satisfazem às premissas levantadas para o bom funcionamento do equipamento. Portanto, a maior corrente entre os dois cenários será selecionada, e, por cima deste valor, será considerado um fator de segurança igual a 1,7, o que nos leva a uma corrente de projeto aproximadamente igual a 15 A.

Dessa forma, a fonte de alimentação proposta será interligada à rede e proverá uma saída CC regulada igual a 12 Volts e 15 A de saída, portanto, terá uma potência de 180 W. As cargas que exigirem uma tensão específica contarão com um regulador de tensão. Como os componentes eletrônicos não demandam corrente alternada, a solução não tem a necessidade de englobar um inversor de corrente. As conexões da fonte serão combinadas com o sistema de alimentação alternativo previsto pela bateria. 

\subsection{Retificador CA/CC}
%interface entre a carga e a rede elétrica

%retificadores passivos (não controlados) 
%Retificador ativo com alto fator de potência

Conversores CA/CC, ou retificadores, convertem uma tensão alternada em uma tensão de saída contínua estável. Os retificadores são classificados de acordo com a capacidade de ajuste do valor da tensão de saída (controlados ou não controlados), o número de fases da tensão de entrada (monofásico, trifásico,etc.) e de acordo com os elementos retificadores. Os retificadores não-controlados utilizam diodos como elementos de retificação. O retificador com ponte de onda completa (\textit{full bridge}), topologia mais aplicada em equipamentos na prática, utiliza quatro diodos para converter os ciclos positivos e negativos para o lado CC antes de estabilizar a tensão de saída com um capacitor \cite{Conversores,Conversores2}. 

Portanto, a etapa de entrada da fonte é baseada em um conversor CC/CC com ponte de onda completa, a configuração está representada na figura \ref{fig:energia_retificador}, onde $L_{ac}$ e $R_{ac}$ são, respectivamente, a indutância e resistência da impedância da fonte alternada de entrada, $D_{1}$, $D_{2}$, $D_{3}$ e $D_{4}$ formam o retificador monofásico de onda completa, e $C_{o}$ é o filtro capacitivo de saída. Suas vantagens e, portanto, a justificativa para a escolha da topologia, são \cite{retificador}:

\begin{itemize}
    \item Em relação aos retificadores de meia onda, possuem maior eficiência de retificação;
    
    \item Possuem baixa perda de energia, pois nenhum sinal de tensão é perdido no processo de conversão;
    
    \item Não necessita de transformador com derivação central;
    
    \item A corrente na carga flui em um único sentido;
\end{itemize}

%\begin{figure}[H]
%\centering
%    \includegraphics[width=0.8\textwidth]{figuras/Energia_Full_bridge.PNG}
%    \caption{Configuração do retificador com ponte de onda completa a ser implementado com indicação do fluxo de corrente (setas vermelhas). Fonte: \citeonline{Conversores}, com modificações do autor.}
%    \label{fig:energia_retificador}
%\end{figure}

\begin{figure}[H]
\centering
\subfloat[][Configuração do retificador com ponte de onda completa a ser implementado com indicação do fluxo de corrente (setas vermelhas).]{
\includegraphics[width=0.5\textwidth]{figuras/Energia_Full_bridge.PNG}
\label{fig:energia_retificador}}
\qquad
\subfloat[][Principais formas de onda do retificador com ponte de onda completa.]{
\includegraphics[width=0.4\textwidth]{figuras/retificador_onda.PNG}
\label{fig:subfig_ondas}}
\caption{Configuração e formas de onda típicas do retificador de onda completa. Fonte: \citeonline{Conversores}, com modificações do autor.}
\label{fig:globfig}
\end{figure}

A cada meio período da tensão alternada de entrada há um par de diodos conduzindo. Dessa forma, modelando a carga como sendo resistiva e observando as formas de onda típicas (figura \ref{fig:subfig_ondas}), podemos descrever a operação do retificador nas seguintes etapas \cite{retificador}:

\begin{itemize}
    \item  No semiciclo positivo os diodos $D_{1}$ e $D_{4}$ estão conduzindo, os diodos $D_{2}$ e $D_{3}$ estão bloqueados e o capacitor $C_{o}$ é carregado até o valor máximo da tensão de entrada;
    
    \item A partir do momento que a tensão pulsante é reduzida, ficando menor que a tensão no capacitor, este começa a descarregar, fornecendo energia à carga. Nesta etapa, o capacitor não se descarrega completamente;
    
    \item No semiciclo negativo os diodos $D_{2}$ e $D_{3}$ estão conduzindo, os diodos $D_{1}$ e $D_{4}$ estão bloqueados e o processo de carregamento do capacitor $C_{o}$ começa novamente. Dessa forma, cerca de metade da energia do capacitor é descarregada e a corrente flui para a carga em uma única direção;
    
\end{itemize}

%\begin{equation}
 % f(x)=\begin{cases}
 %   1, & \text{if $x<0$}.\\
 %   0, & \text{otherwise}.
 % \end{cases}
%\end{equation}

O objetivo da configuração é garantir uma tensão de saída CC constante. A configuração proposta garantirá uma alimentação bivolt e sua saída é responsável por alimentar a etapa do conversor CC/CC da fonte proposta.

\begin{table}[H]
    \centering
    \footnotesize
    \caption{Parâmetros de projeto do retificador monofásico com ponte de onda completa.}
    \label{retificador}
    \begin{adjustbox}{max width = \textwidth}
        \begin{tabular}{|l|c|c|}
            \hline
            \rowcolor[HTML]{A8DADC}
            \textbf{Parâmetro} & \textbf{Simbologia} & \textbf{Valor}  \\ \hline
            Tensão eficaz de Entrada & $V_{CA}$ & 127/220
            \\ \hline
           % Tensão máxima de Entrada* & $V_{CAmax}$ & 133/231
            %\\ \hline
            %Tensão mínima de Entrada* & $V_{CAmin}$ & 117/202
           % \\ \hline
              Potência de saída & $P_{o}$	 & 200
             \\ \hline
              Frequência da Rede & $f_{r}$	 & 60Hz 
             \\ \hline
            %  Frequência de Chaveamento & $f_{s}$	 & -Hz
             %\\ \hline
           %     Tensão de Saída & $v_{out}$ & -
            % \\ \hline
            %  Indutor e resistor do \textit{Boost} & L e R & -
            % \\ \hline
           % Impedância da fonte & $ L_{ac}$, $R_{ac}$ &- 
            % \\ \hline
             %   Ondulação da Corrente no Indutor & $\Delta I_{L}$ &- 
            % \\ \hline
               Ondulação da Tensão no Capacitor (\textit{ripple} máximo) & $\Delta V_{C}$ & 10\%
             \\ \hline
                Rendimento & $\eta$ & 0,9
             \\ \hline
            %    fator de ciclo (\textit{duty cycle}) & $D$ & -
             %\\ \hline
            % \multicolumn{3}{l}{*Valores baseados na tabela 4 (Anexo 1)
 %do módulo 8 do Prodist.} \\
              \multicolumn{3}{l}{NOTAS: Este design é baseado em valores típicos.} \\ 
        \end{tabular}
    \end{adjustbox}
\end{table}

A tabela \ref{retificador_componentes} contém os componentes selecionados para o projeto do retificador. O apêndice \ref{Energia_memorial} contém o memorial de cálculo detalhado do projeto.

\begin{table}[H]
    \centering
    \caption{Componentes selecionados.}
    \label{retificador_componentes}
    \begin{adjustbox}{max width = \textwidth}
        \begin{tabular}{|c|c|c|}
        %{|L{7cm}|C{3cm}|C{3cm}|C{2cm}|C{1cm}|}
            \hline
            \rowcolor[HTML]{A8DADC}
            \textbf{Componente} & \textbf{Variável do circuito} & \textbf{Modelo/ Valor Comercial}\\ \hline
            \multirow{2}{*}{Capacitor}& Capacitância & 200uF\\ \cline{2-3} 
            & Tensão de Trabalho & 400V
            \\ \hline 
            \multirow{3}{*}{Diodo}& Modelo & 1N5404 \\ \cline{2-3}
            & Corrente média & 3A \\ \cline{2-3} 
            & Tensão reversa máxima & 400V
             \\ \hline 
        \end{tabular}
    \end{adjustbox}
\end{table}

%Porém, deve-se o impacto na qualidade da energia elétrica, pois retificadores a diodos proporcionam baixo fator de potência e geram harmônicos de corrente para a rede de alimentação. Em relação às normas e recomendações para as limitações de injeção de harmônicas tem-se as estabelecidas pelo \textit{International Eletrotechnical Commission - IEC} e \textit{Institute of Electrical and Electronic Engineers} - IEEE, que impõem limites máximos de harmônicas de corrente e de tensão, e o módulo 8 dos Procedimentos de Distribuição de Energia Elétrica no Sistema Elétrico Nacional (PRODIST), que dispõe sobre a qualidade da energia elétrica e impõe limites máximos apenas de harmônicas de tensão \cite{retificador1}.
%tabela com os valores TDH máximos e FP admitidos
%distorção harmônica total (total harmonic distortion - TDH)

%Como soluções propostas para aumentar o fator de potência e diminuir a distorção harmônica em conversores estáticos, tem-se dois principais métodos: um que utiliza componentes passivos e outro de componentes ativos. As técnicas de correção passiva utilizam indutores e capacitores, porém, apesar da simplicidade, robustez, fácil implementação e operação, possuem a desvantagem de ter elevado peso, volume e custo \cite{Conversores1}.

%A tecnologia de eletrônica de potência possibilita uma solução mais compacta, conhecida como de solução ativa, que utiliza interruptores controlados. Esta solução apresenta custos menores e maior eficiência, porém, são mais complexas que as técnicas de solução passiva. Dentre as técnicas dessa solução tem-se a associação de um estágio para correção do fator de potência a retificadores não-controlados, que são os circuitos de correção de fator de potência (\textit{Power Faction Correction - PFC's}), que garantem uma corrente de entrada senoidal com um TDH abaixo de 5\%, assim como um fator de potência unitário \cite{Conversores}

%\subsubsection{Conversor \textit{Boost}}

%O conversor \textit{Boost} se destaca atuando como pré-regulador por causa de sua eficiência, simplicidade e fácil obtenção da corrente de entrada com baixa distorção. Na figura \ref{fig:energia_etapa1} pode-se visualizar o esquema do conversor AC/CC que será projetado para o dispensador Pill Watcher. Esta tipologia tem a característica de ter baixo volume, não ser obrigatório ter um transformador acoplado e por utilizar apenas um interruptor, o que torna o controle e a modulação bem simples \cite{retificador1}.

%\begin{figure}[H]
 %   \centering
  %  \includegraphics[width=1\textwidth]{figuras/estágio de entrada.png}
   % \caption{Etapa de entrada da fonte proposta. Fonte: Autor.}
    %\label{fig:energia_etapa1}
%\end{figure}

%Suas vantagens são:

%\begin{itemize}
 %   \item O indutor na entrada ($L_{boost}$) tem a função de absorver as variações na tensão, fazendo com que o restante do circuito não seja atingido;
    
 %   \item O capacitor de saída ($C_{b}$), que tem a função de armazenar a energia, por operar em alta tensão, permite valores menores de capacitores;
    
 %   \item O controle da forma de onda é continuado para todo valor da tensão de entrada instantâneo;
    
  %  \item O transistor ($S_{boost}$) possui um acionamento simples, que pode ser feito a partir de um sinal de baixa tensã referenciado ao terra;
    
%\end{itemize}

%E suas desvantagens \cite{retificador1}:

%\begin{itemize}
%    \item Em qualquer etapa de operação há três semicondutores conduzindo corrente ao mesmo tempo (dois diodos do retificador, o interruptor ($S_{boost}$) ou o diodo de saída ($D_{boost}$)), o que pode prejudicar o rendimento do conversor;
    
%    \item Toda a potência está distribuída apenas entre o interruptor ($S_{boost}$) e o diodo de saída ($D_{boost}$), dessa forma, a densidade de potência nos dois componentes é maior, exigindo, assim, dispositivos mais caros.
    
  %  \item O conversor CC/CC da etapa posterior deve operar com uma tensão de entrada relativamente elevada;
    
 %   \item A isolação entre a entrada e a saída não é possível;
    
%\end{itemize}

%O retificador com pré-regulador \textit{Boost} PFC pode ser descrito como tendo dois principais estágios: um retificador monofásico passivo em ponte completa e um conversor cc/cc \textit{Boost}. A configuração proposta trabalhará no modo de condução descontínua (\textit{discontinuous conduction mode} -DCM), operar no DCM significa dizer que a corrente no indutor chega a zero (energia armazenada chega a zero) a cada período de chaveamento. A frequência de operação é bem maior que a frequência da rede, e  por isso a tensão de entrada é considerada constante. 

%Portanto, desconsiderando o filtro de alta frequência ($L_{f}$ e $C_{f}$), modelando a carga como sendo resistiva e observando as formas de onda típicas (figura \ref{fig:subfig1}), podemos descrever a operação do pré-regulador \textit{Boost} PFC em três etapas: CITE TOP

%\begin{itemize}
  %  \item 1º etapa: representada na figura \ref{fig:subfig2} e referente ao intervalo $\Delta t_{1}$, o transistor ($S_{boost}$) é ligado e o indutor ($L_{boost}$) fica submetido a tensão de entrada, que alimentará o capacitor e a carga quando o transistor for desligado. O diodo ($D_{boost}$) fica reversamente polarizado, pois a tensão de saída é maior que a da entrada. 
    
 %   \item 2º etapa: representada na figura \ref{fig:subfig3} e referente ao intervalo $\Delta t_{2}$, o transistor ($S_{boost}$) é desligado e diodo ($D_{boost}$) entra em condução. O indutor transistor ($L_{boost}$) alimenta, com a energia acumulada da 1º etapa, o capacitor ($C_{b}$) e a carga ($R_{b}$).
    
 %   \item 3º etapa: representada na figura \ref{fig:subfig4} e referente ao intervalo $\Delta t_{3}$, a corrente no indutor ($L_{boost}$) chega a zero, o diodo ($D_{boost}$) fica polarizado reversamente e o capacitor ($C_{b}$) alimenta a carga ($R_{b}$).
    
%\end{itemize}

%O transistor ($S_{boost}$) é controlado utilizando modulação por largura de pulso (\textit{Pulse Width Modulation} - PWM) e frequêcia constante, o que faz com que o valor de pico da corrente do indutor ($L_{boost}$) seja diretamente proporcional â tensão de entrada. O filtro de alta frequência ($L_{f}$ e $C_{f}$) elemina as variações bruscas da corrente no indutor ($L_{boost}$), tornando-a senoidal. 

%\begin{figure}[h]
%\centering
%\subfloat[][Formas de onda do conversor no modo de condução descontínua.]{
%\includegraphics[width=0.5\textwidth]{figuras/retificador_onda.PNG}
%\label{fig:subfig1}}
%\qquad
%\subfloat[][1º etapa de operação: intervalo $\Delta t_{1}$.]{
%\includegraphics[width=0.5\textwidth]{figuras/1etapa_boost.png}
%\label{fig:subfig2}}
%\subfloat[][2º etapa de operação: intervalo $\Delta t_{2}$.]{
%\includegraphics[width=0.5\textwidth]{figuras/2etapa_boost.png}
%\label{fig:subfig3}}
%\qquad
%\subfloat[][3º etapa de operação: intervalo $\Delta t_{3}$.]{
%\includegraphics[width=0.5\textwidth]{figuras/3etapa_boost.png}
%\label{fig:subfig4}}
%\caption{Etapas de operação do retificador pré-regulador de fator de potência (PFP) proposto.}
%\label{fig:globfig}
%\end{figure}

%Modo de Condução Contínua (\textit{Continuous Conduction Mode} - CCM)

%O ripple (ondulação residual) de tensão resultante  
%Uma solução para a redução das perdas nessa configuração é adicionar um Transistor de Efeito de Campo de Óxido de Metal Semicondutor (MOSFET) em paralelo a cada diodo. - não será implementado 

%\begin{itemize}
   % \item \textit{Indutância ($L_{boost}$)}
    
 %   O fator de ciclo máximo é limitado de acordo com a equação \ref{dmax}:
    
 %   \begin{equation}
 %       D_{max} = \frac{D - 1}{D}
 %       \label{dmax}
 %   \end{equation}
    
 %   A indutância ($L_{boost}$) é calculada de acordo com a equação 
    
 %   \begin{equation}
  %      L_{boost} = \frac{V_{B}^{2} \cdot D^{2} \cdot 0,48}{2 \cdot M \cdot f_{s} \cdot P_{boost} \cdot (M-0,92)}
   %     \label{dmax}
%    \end{equation}
    
 %   O transistor ($S_{boost}$) e o diodo ($D_{boost}$) são selecionados considerando a suportabilidade da tensão reversa máxima (que é igual a tensão do barramento CC) e 
    
%\end{itemize}

\subsection{Conversor CC/CC}

Conversores CC/CC, ou \textit{choppers}, convertem a corrente contínua para uma corrente contínua em um nível de tensão diferente. A entrada para este tipo de conversor é uma tensão não regulada, portanto, seu objetivo é produzir uma tensão de saída regulada que seja adequada à carga. Na prática, é possível alcançar uma eficiência de 70\% a 95\%. O controle e regulação da tensão de saída se dá por meio da modulação por largura de pulso (\textit{Pulse-width modulation} - PWM). São utilizados dispositivos semicondutores como diodos, transistores de efeito de campo metal-óxido-semicondutor (MOSFET), transistores bipolares de porta isolada (IGBT), transistores bipolares de junção (TJB) ou tiristores. As frequências de comutação típicas estão na faixa de 1kHz a 1MHz, a depender da velocidade dos dispositivos semicondutores \cite{forward}.

Portanto, o conversor CC/CC produz uma tensão CC de saída cuja magnitude é controlada por meio do fator de ciclo (\textit{duty cycle}), usando circuitos auxiliares. 
%A taxa de conversão M(D) é definida como sendo a razão entre a tensão de saída CC e a tensão de entrada CC sob condições de estado estacionário \cite{forward}.

  %  \begin{equation}
   %     M(D) = \frac{V}{V_{g}}
%        \label{dmax}
 %   \end{equation}

\subsubsection*{Conversor \textit{Forward}}

Selecionar a topologia errada pode resultar em um projeto de design que não atende às suas metas de custo, metas de eficiência ou uma série de outros requisitos que se é necessário ter. Tem-se vários circuitos/topologias que podem aumentar ou diminuir a magnitude da tensão de saída CC e/ou inverter sua polaridade. Em muitas aplicações, é desejado incorporar um transformador no conversor de comutação para obter uma isolação entre os circuitos de entrada e de saída, evitando, assim, choques elétricos. O tamanho e o peso do transformador variam inversamente com a frequência do sistema, portanto, as altas frequências levam a grandes reduções no tamanho do transformador \cite{forward}.

Há várias maneiras de incorporar um transformador isolador em um conversor CC/CC. O conversor isolado tipo \textit{Forward} (\textit{Buck} isolado) com um único semicondutor é usualmente usado em fontes off-line na faixa de potência até 200W e com uma alta corrente de saída, sua configuração pode ser vista na figura \ref{fig:energia_forward}, onde o primário do transformador é colocado em série com um transistor ($Q_{1}$) e o enrolamento $n_{2}$ é colocado em série com o diodo $D_{1}$ a fim de desmagnetizar o núcleo do transformador ao final de cada período de comutação. O secundário do transformador é conectado a um filtro LC de saída e o diodo $D_{2}$ é inserido de forma a prevenir uma corrente negativa no secundário do transformador  \cite{Conversores}.

\begin{figure}[H]
\centering
    \includegraphics[width=0.8\textwidth]{figuras/Energia_forward.PNG}
    \caption{Configuração do conversor  \textit{Forward} a ser implementado com indicação do fluxo de corrente (setas vermelhas). Fonte: \citeonline{forward}, com modificações do autor.}
    \label{fig:energia_forward}
\end{figure}

Suas vantagens e, portanto, a justificativa para a escolha da topologia, são \cite{Conversores}:

\begin{itemize}
    \item Isolação galvânica entre a tensão de entrada e de saída, conferindo segurança ao sistema e evitando riscos de choques elétricos no momento da limpeza do equipamento. 
    
    \item Não depende de armazenamento de energia, a transferência de ocorre diretamente através do transformador;
    
    \item Corrente de saída é de boa qualidade;
    
\end{itemize}

Dessa forma, podemos descrever a operação do retificador nas seguintes etapas:

\begin{itemize}
    \item  O transistor $Q_{1}$ conduz, junto ao diodo $D_{2}$, enquanto os diodos $D_{1}$ e $D_{3}$ estão bloqueados, situação esta em que não há corrente no enrolamento de magnetização;
    
    \item O transistor $Q_{1}$ fica bloqueado, ao passo que o diodo $D_{3}$ conduz a corrente de carga e o diodo de desmagnetização $D_{1}$ conduz a energia de volta para a fonte;
    
    \item  A corrente armazenada no indutor chega a zero, cessando a devolução de energia à fonte enquanto o diodo $D_{3}$ continua conduzindo; 
    
\end{itemize}

\begin{table}[H]
    \centering
    \footnotesize
    \caption{Parâmetros de projeto do conversor \textit{Forward}.}
    \label{Conversor_cc/cc1}
    \begin{adjustbox}{max width = \textwidth}
        \begin{tabular}{|l|c|c|}
            \hline
            \rowcolor[HTML]{A8DADC}
            \textbf{Parâmetro} & \textbf{Simbologia} & \textbf{Valor}  \\ \hline
            Tensão de Entrada & $V_{i}$ & 127-220 Vcc
            \\ \hline
           % Tensão mínima de Entrada* & $V_{CAmin}$ & 117/202 V
            %\\ \hline
            % Corrente de entrada & $I_{i}$ & X A
            %\\ \hline
            Tensão de saída & $V_{o}$ & 12 V
            \\ \hline
            Corrente de saída & $I_{o}$ & 15 A
            \\ \hline
              Potência de saída & $P_{o}$ & 200 W 
             \\ \hline
              Frequência de Chaveamento & $f_{s}$ & 50 kHz
             \\ \hline
             %   Mínima carga & $L_{min}$ & 10 \%
             %\\ \hline
               Ondulação da Tensão de entrada (\textit{ripple}) & $\Delta V_{i(ripple)}$ & 5\%
             \\ \hline
                Ondulação da Corrente de saída (\textit{ripple}) & $\Delta I_{out(ripple)}$ & 5\%
             \\ \hline
                Rendimento & $\eta$ & 0,9
             \\ \hline
                fator de ciclo (\textit{duty cycle}) & $D$ & 0,45
             \\ \hline
                Densidade de corrente & $J$ & 450 $A/cm^{2}$
             \\ \hline
                 Fator de ocupação do primário & $k_{p}$ & 0,5
             \\ \hline
                 Fator de ocupação da área do enrolamento & $k_{w}$ & 0,4
             \\ \hline
                 Indução magnética & $\Delta B$ & 0,18
             \\ \hline
                  Queda de tensão no diodo & $V_{f}$ & 0,7
             \\ \hline
             \multicolumn{3}{l}{NOTAS: Este design é baseado em valores típicos.} \\ 
           % \multicolumn{3}{l}{*Valores baseados na tabela 4 (Anexo 1)
 %do módulo 8 do Prodist.} \\
        \end{tabular}
    \end{adjustbox}
\end{table}

\subsubsection*{Circuitos integradores}

Como a tensão de saída $v(t)$ do conversor é uma função do \textit{duty cycle}, um sistema de controle pode ser construído de forma que varie o \textit{duty cycle} para fazer com que a tensão de saída siga uma determinada referência, a operação deste sistema se dá da seguinte maneira: a tensão de saída é detectada usando um divisor de tensão, que então é comparada com uma tensão CC de referência. O sinal de erro resultante é passado por uma rede de compensação em um amplificador operacional (Amp-Op), resultando em uma tensão analógica que, em seguida, é alimentada por um modulador por largura de pulso (PWM) \cite{forward}.

O modulador produz uma forma de onda de tensão comutada que controla a porta do dispositivo semicondutor. Dessa forma, o \textit{duty cycle} é proporcional à tensão analógica de controle. Essa abordagem é chamada de controle do modo de tensão. Se esse sistema de controle for bem projetado, o \textit{duty cycle} é ajustado automaticamente, de modo que a tensão de saída do conversor siga a tensão de referência e seja independente das variações da tensão de entrada ou da corrente de carga \cite{forward}.

O circuito integrador de controle PWM provê o \textit{Duty Cycle} e os elementos para implementar o controle do conversor CC/CC. O \textit{Duty Cycle} é gerado a partir de uma comparação entre o sinal triangular e sinal contínuo. Optou-se por utilizar o circuito integrado TL494 da \textit{Texas Instruments}, que é versátil e faz um ótimo trabalho ao englobar todas as etapas necessárias para o controle em um único circuito integrado.

A frequência de operação do TL494 será de 50kHz. A frequência de trabalho do CI deve ser projetada para ser o dobro da desejada, portanto, $2 \cdot 50kHz = 100kHz$. Observando o gráfico da figura \ref{fig:chave}, utilizaremos uma capacitância de 1nF e uma resistência de $12k\Omega$, dessa forma, tem-se 50kHz. 

\begin{figure}[H]
\centering
\subfloat[][Gráfico da frequência de oscilação \textit{versus} resistência e capacitância do CI TL494. Fonte: \textit{Datasheet} do CI TL494.]{
\includegraphics[width=0.4\textwidth]{figuras/chaveamento.PNG}
\label{fig:chave}}
\qquad
\subfloat[][Gráfico da corrente no coletor \textit{versus} corrente direta no diodo. Fonte: \textit{Datasheet} do CI TIL111.]{
\includegraphics[width=0.4\textwidth]{figuras/isolar.PNG}
\label{fig:isolador}}
\caption{Gráficos dos CI's utilizados.}
%\label{fig:globfig}
\end{figure}

A isolação da realimentação do circuito é feita por meio do CI TIL111, para saber o resistor que vai em série com o diodo emissor foi utilizado o gráfico presente na imagem \ref{fig:isolador}. A tabela \ref{Conversor_cc/cc2} contém os componentes selecionados para o projeto da fonte chaveada. O apêndice \ref{Energia_memorial} contém o memorial de cálculo detalhado para o projeto e o diagrama completo da fonte, figura \ref{fig:diagrama_fonte1}.

\begin{table}[H]
    \centering
    \footnotesize
    \caption{Componentes selecionados para o conversor \textit{Forward.}}
    \label{Conversor_cc/cc2}
    \begin{adjustbox}{max width = \textwidth}
        \begin{tabular}{|l|c|}
            \hline
            \rowcolor[HTML]{A8DADC}
          %  \multicolumn{4}{2}{Enrolamentos}  \\ \hline
            Parâmetro & Modelo/Valor Comercial
            \\ \hline
            Diodo de desmagnetização &  MUR110EG
            \\ \hline
            Diodo Série Saída &  MUR860G
            \\ \hline
            Diodo Paralelo Saída & MUR860G
             \\ \hline
              MOSFET & IRFBG20
             \\ \hline
              Capacitor & 220uF
             \\ \hline
              Indutor & 100uH
             \\ \hline
              Circuito de controle & CI TL494
             \\ \hline
              \textit{Feedback} Isolado & CI TIL111
             \\ \hline
              Núcleo do transformador & E-42/21/15
             \\ \hline
        \end{tabular}
    \end{adjustbox}
\end{table} 

\subsection{Sistema Autônomo de Emergência}

Nas situações em que ocorrer a interrupção na rede elétrica da concessionária, a carga do projeto será instantaneamente alimentada pelo sistema de emergência, de forma a evitar o desligamento brusco do equipamento e garantir seu funcionamento por um tempo determinado. 

A fonte primária do sistema de emergência será provida por uma bateria (acumulador de energia) e seu funcionamento se dará de maneira que no regime de operação normal, em que as cargas são alimentadas pela fonte, a bateria estará fora de serviço. Neste mesmo regime de operação, a bateria será carregada pela fonte principal por meio de um carregador de bateria. Haverá a comutação instantânea para a operação normal assim que a entrada de tensão alternada (CA) seja normalizada.

%A carga da bateria será garantida por meio de um 
%conversor bidirecional CC
%controlador de carga

A tensão nominal da bateria será igual a tensão de alimentação da carga, que foi delimitada como sendo 12 V, e os seguintes itens foram considerados para o seu dimensionamento \cite{bateria}:

\begin{itemize}
    \item Autonomia do sistema, ou seja, o tempo mínimo que a bateria deverá prover para que o equipamento funcione nos cenários emergenciais é de 1 hora;
    \item O custo da bateria, este aumenta quanto maior for o tempo de autonomia.
    \item Profundidade de descarga. A vida útil da bateria é reduzida quanto maior for a profundidade de descarga.
\end{itemize}

Ao considerar uma autonomia da bateria (o tempo mínimo que a bateria deverá prover), de acordo com a corrente de projeto estabelecida em \ref{Solução_energia} (15A), de uma hora. 

Será utilizada uma Bateria de Lítion Íon modelo CRG-01203M, considerando carga suficiente para a disponibilidade no mercado e autonomia desejada. A escolha dessa bateria deu-se pela alta densidade de energia e elevado potencial, baixo índice de auto-descarga e elevadas correntes de descarga. Optou-se pela opção de apenas uma bateria no dispensador, para não alterar as dimensões e melhor atender o projeto. 

A bateria escolhida para alimentar o projeto, tem as seguintes especificações: Capacidade de carga 15Ah, tensão nominal 12 V, corrente de carga normal 2A e corrente de descarga contínua normal de 7,5A.

\begin{figure}[H]
\centering
    \includegraphics[width=0.4\textwidth]{figuras/bateria.PNG}
    \caption{Bateria de Íon-Lítio, 12V 15Ah selecionada para o projeto.}
   % \label{fig:energia_forward}
\end{figure}

O acionamento do circuito de emergência se dá por meio de uma chave (relé), que tem a função de mudar a carga para o circuito alimentado pela bateria quando houver falha na fonte chaveada. O relé selecionado foi o relé reversor selado N.A/N.F – 15A/12V, de 5 Terminais.

As proteções são essenciais para reduzir a probabilidade e as consequências das falhas. Apesar de apresentarem um custo a mais no projeto, as proteções acrescentam confiabilidade ao sistema. Para a proteção da bateria, foi utilizado um fusível de vidro 5x20 - 2,5A 250V, que se abre no caso de uma corrente excessiva. Por meio de uma verificação periódica será possível observar o status da bateria no mostrador de cristal líquido (\textit{Liquid Crystal Display - LCD}) do equipamento.

\section{Dimensionamento dos Condutores}

%tabela 47 e 38 NBR 5410

O dimensionamento dos condutores foi realizado com base na norma NBR 5410/2004. O objetivo principal desta norma é estabelecer condições a que devem satisfazer as instalações elétricas de baixa tensão a fim de garantir a segurança de pessoas e animais, o funcionamento adequado da instalação e conservação dos bens presentes no local. 

\begin{enumerate}
    \item Método 1: Seção referente a uma análise das condições apresentadas pela norma, de acordo com o tipo de linha e utilização do circuito. 
    
    \item Método 2: Seção referente a escolha da corrente nominal do projeto, em que para cada circuito terá uma seção específica de condutor. 
\end{enumerate}

De acordo com a norma, os métodos de referências são os métodos de instalação para os quais a capacidade de condução de corrente foi determinada por ensaio ou por cálculo. A referência utilizada no projeto foi o F: cabos unipolares justapostos ao ar livre. Para fins de dimensionamento dos condutores, são considerados os seguintes circuitos:

\begin{itemize}
    \item Circuito A: entrada da rede;
    
    \item Circuito B: saída da fonte/bateria;
    
    \item Circuito C: motores de passo;
    
    \item Circuito D: atuador elétrico;
    
    \item Circuito E: atuador \textit{push pull};
    
    \item Circuito F: motor DC (esteira);
    
    \item Circuito G: componentes eletrônicos;
\end{itemize}

De acordo com a tabela 47 da NBR 5410/2004, para o método 1, os circuitos de tomadas de corrente são considerados circuitos de força, portanto, os circuitos A, B, C, D, E, F e G são considerados de força. O método 2 depende da corrente de projeto do circuito, de forma que a capacidade de condução de corrente dos condutores deve ser igual ou superior à corrente de projeto. A corrente de projeto ($I_{projeto}$) é calculada de acordo com a equação \ref{energia_corrente1}, onde é considerada um rendimento ($\eta$) de 100\%, um fator de potência (FP) unitário e queda de tensão inferior a 4\%. A seção final dos condutores é dada pela maior seção entre os dois parâmetros encontrados.
    
    \begin{equation}
        I_{projeto} = \frac{P_{circuito} \cdot 1,2}{V \cdot FP \cdot \eta}
        \label{energia_corrente1}
    \end{equation}
    %\left(  \right)}

\begin{table}[H]
    \centering
    \footnotesize
    \caption{Determinação da seção dos condutores.}
    \label{energia_seção}
    \begin{adjustbox}{max width = \textwidth}
        \begin{tabular}{|c|c|c|c|c|c|}
            \hline
            \rowcolor[HTML]{A8DADC}
          %  \multicolumn{4}{2}{Enrolamentos}  \\ \hline
            Circuito & Potência (W) & Corrente de projeto (A) & Método 1 ($mm^{2}$)& Método 2 ($mm^{2}$)& Seção escolhida ($mm^{2}$)
            \\ \hline
            A  & 222 & 1,21 & 2,5 & 0,5 & 2,5
            \\ \hline
             B & 180 & 18 & 2,5 & 1,5 & 2,5
            \\ \hline
             C & 4,8 & 0,4 & 2,5 & 0,5 & 2,5
            \\ \hline
             D & 24 & 2 & 2,5 & 0,5 & 2,5
            \\ \hline
             E & 5,6 & 1,12 & 2,5 & 0,5 & 2,5
            \\ \hline
             F & 19,2 & 1,6 & 2,5 & 0,5 & 2,5
            \\ \hline
             G & 74,34 & 7,43 & 2,5 & 0,5 & 2,5
            \\ \hline
        \end{tabular}
    \end{adjustbox}
\end{table}
% 1 motor de passo 0,5mm²

%O primeiro parâmetro é uma análise das condições apresentadas pela norma, de acordo com o tipo de linha e utilização do circuito. Como o circuito geral do dispensador contem motores e componentes eletrônicos, a utilização do circuito é classificada como de força. Como o sistema de alimentação apresenta uma fonte que mudará a tensão de entrada,o dimensionamento dos condutores foi realizado para os cabos de entrada rede elétrica (220V), saída da fonte (12V), para os motores (circuito A) e equipamentos eletrônicos(circuito B). 

%A seção final dos condutores é dada pela maior seção entre os dois parâmetros encontrados. Desse modo, de acordo com a tabela 47 da NBR 5410/2004 tem-se que a seção mínima dos condutores de entrada, para os circuitos A é Ymm(Cu) e para o circuito B, Xmm. 

\section{Sistema de proteção}

O fusível é um dispositivo de proteção contra sobrecorrente que opera quando houver uma sobrecarga ou curto-circuito no sistema, evitando, assim, danos no isolamento dos condutores e/ou nos componentes do \textit{Pill Watcher}. Este deve ser dimensionado para uma corrente no mínimo 20\% maior que a corrente de operação do circuito que irá proteger.

    \begin{equation}
        I_{\text{fusível}} = 1,2 \cdot \frac{P_{\text{operação}}}{V_{N} \cdot FP \cdot \eta} = \frac{222 \cdot 1,2}{110 \cdot 1 \cdot 1} = 2,42 \quad A
        \label{energia_corrente2}
    \end{equation}
    %\left(  \right)}
    
Para o cálculo, equação \ref{energia_corrente2}, foi considerado a tensão de entrada igual a 110 pois seria a situação mais crítica. A categoria de utilização do fusível será gG para a proteção de linha que atuará na presença tanto de curto-circuito como de sobrecarga. Foi selecionado um fusível de vidro 5x20 - 2,5A 250V. 

\section{Barramentos CC}

As cargas que exigem tensão especifica no projeto são: microcontroladores, raspberry, \textit{display} e sensores fotoelétricos e de biometria, todos em um nível de 5V. Somada as correntes das cargas, obteve-se 4,015A. Portanto, necessita-se de um conversor abaixador de tensão (\textit{Step Down}), optou-se pelo módulo regulador de tensão XL4016, com entrada de 4-40V até 8A e saída de 1.25V A 36V. 

\begin{figure}[H]
\centering
    \includegraphics[width=0.4\textwidth]{figuras/barramento_5V.PNG}
    \caption{Conversor CC/CC \textit{Step-Down} utilizado para transformar a tensão de 12V para 5V.}
   % \label{fig:energia_forward}
\end{figure}
