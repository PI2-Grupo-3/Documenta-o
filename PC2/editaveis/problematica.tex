\chapter[Problemática]{Problemática}

Conforme a Resolução nº 304, de 17 de setembro de 2019, todas as partes envolvidas na produção, armazenamento, distribuição e transporte devem se responsabilizar pela qualidade e segurança dos medicamentos \cite{RDC_304}. Entretanto, muitas instituições de longa permanência improvisam as instalações de seus locais de armazenamento de medicamentos,  e muitas vezes se esquecem da Política de Boas Práticas de Distribuição e Armazenamento (BPDA), uma vez que, este espaço deve garantir a qualidade, a segurança e o controle dos produtos estocados.

A estabilidade do medicamento depende de alguns fatores intrínsecos, que estão relacionados aos processos produtivos e características específicas do produto, enquanto que os fatores extrínsecos, estão associados às condições ambientais, as quais estão diretamente relacionados com o armazenamento, como por exemplo, luminosidade, temperatura e umidade \cite{Souza_2018}. À vista disso, sabe-se que alterações na estocagem podem levar a perda total ou parcial das propriedades medicamentosas, podendo chegar até o estágio de toxicidade maior que a do produto original.


Por outro lado, tem-se as adversidades em clínicas de repouso, onde muitas vezes um cuidador é responsável por vários idosos, com tratamentos e medicações diferentes, o que dificulta ministrar medicamentos, principalmente quando o paciente necessita ingeri-los em diferentes horários do dia, situação essa bastante comum à boa parte da população idosa, o que consequentemente propicia o aumento de erros.


Assim, a partir da análise dos erros, ocorridos nos Estados Unidos pela FDA (MedWatch Program) e USP-ISMP (Medication Errors Reporting Errors), mostra que as causas dos erros são multifatoriais. Os erros mais frequentes estão relacionados à dose, hora e frequência erradas, omissão, falhas na interação com outros serviços, violação de regras e medicamento errado, o que pode causar prejuízos à saúde do idoso e provocar efeitos colaterais diversos, além de não ocasionar os resultados esperados do tratamento \cite{Freire_2009}. Além do mais, o ato de administrar medicamentos é a última oportunidade de intervir evitando a ocorrência de erros, os quais podem ter sido iniciados no processo de fabricação \cite{Azevedo_2011}.

\section{Justificativa}

O projeto tem como base implementar mecanismos que reduzam ou impeçam o número crescente de erros de armazenamento e administração de medicamentos em instituições de longa permanência. Sendo assim, o grande diferencial desse dispositivo em relação aos já existentes no mercado é a opção de realizar o gerenciamento no controle da medicação para mais de um paciente. Consequentemente, a partir dessa iniciativa  é possível assegurar os denominados “nove certos”: paciente certo, medicamento certo, dose certa, via certa, hora certa, tempo certo, validade certa, abordagem certa e registro certo.

% escrever mais

\section{Objetivos}
\subsection{Objetivos Gerais}

%O projeto PillWatcher consiste na concepção de um gerenciador de medicamentos para idosos que residem em instituições de longa permanência ou clínicas geriátricas, por meio de um dispensador automático de comprimidos, o qual armazena, separa as dosagens de medicações e as disponibiliza para os pacientes nos horários previamente prescritos pelo médico e cadastrados no aplicativo.%Essa iniciativa tem como finalidade reduzir a probabilidade de falhas e aumentar a chance de interceptá-las antes de resultar em prejuízo ao paciente. <- texto original
O projeto Pillwatcher consiste na concepção de um dispensador automático de comprimidos, cápsulas e drágeas, o qual irá armazenar, separar dosagens de medicações e disponibilizar para pacientes nos horários previamente prescritos pelo médico e cadastrados no aplicativo além de reduzir probabilidade de falhas e aumentar a chance de interceptá-las antes de resultar em prejuízo ao paciente.


%detalhar sobre o dispositivo e o app (?)

\subsection{Objetivos Específicos} \label{section:Obj_esp}

\begin{itemize}
\item Garantir que o projeto tenha integração entre todas as engenharias;
\item Certificar a usabilidade do dispositivo e aplicativo pelo corpo técnico de médicos, enfermeiros e cuidadores;
\item Facilitar a comunicação entre Homem - Máquina, utilizando IoT para comunicação entre aplicação em software e eletrônica. 
\item Integrar os dados cadastrados presentes no aplicativo com o dispensador de comprimidos;
\item Assegurar que os remédios e as doses devam ser entregues corretamente para cada paciente nos horários indicados;
\item Confirmar que os medicamentos no estoque do dispensador sejam conservados com temperatura controlada entre 15ºC e 30ºC, e umidade relativa de 40\% a 70\%;
\item Automatizar o processo de rotina de distribuição de medicação através de alertas e notificações. Dessa forma, deve-se alertar o corpo técnico de saúde da residência sênior sobre o horário para ministração de medicamentos que não sejam medicamentos sólidos (injetáveis, pomadas, etc) e indicar a disponibilidade da dose medicamentosa de comprimidos, drágeas e capsulas no dispensador;
\item Facilitar a criação de relatórios de verificação de consumo de medicação;
\item Realizar o controle de estoque dos medicamentos sólidos inseridos no dispensador;

\end{itemize}







